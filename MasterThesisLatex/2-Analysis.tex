\chapter{Analysis}

\section{Dataset}

The dataset was provided from the laboratory of XXX.

\section{Technique}

Talk about what Transfer Learning is and why it could help this thesis.

\section{Deeplearning Frameworks}

There are two main Deeplearning Frameworks that can readily be used. Tensorflow which was originally developed by researchers and engineers at Google Brain and is based on Theano and PyTorch that was built by researchers at Facebook. They are both open source and free to use, but the flavor is quite different. While TensorFlow uses static computational graphs that need to be built prior to compilation and run in their run engine, PyTorch uses dynamic computational graphs tthat can be more interpreted than compiled. Programming in PyTorch is much more pythonic whereas in TensorFlow the user needs first to get used to the Tensorflow way of doing things. Like building the whole graph in advance, using placeholders for all weights and variables and then create a session in which the graph can be executed. Debugging in Tensorflow is more difficult since it needs at least two different debuggers to be used. One for the tensors and their values, and one for the python code itself. That makes it much less intuitive to simple debug the underlying code while keeping track of all tensors. In PyTorch the native debugger may be used for the whole codebase, including all the variables and weights. Data parallelisme is much easier to use in PyTorch since the distribution of the code and data onto all the GPU's happens automagically. Whereas in Tensorflow much more manual work and careful thought needs to be applied to achieve the same behaviour. Since PyTorch is one big framework it gives more the feeling of working with one framework that uses a very pythonic way to handle things. TensorFlow on the other hand is more like an aggregation of many libraries that work together to achieve a common goal. Although that was the case in early 2018 things are changing very fast for these frameworks. Tensorflow was much better for production environments but with PyTorch version 1.0 this advantage is closing fast. Table \ref{tbl:DeepLearningFrameworks} summarizes the different qualities of  both frameworks at the start of the Masterthesis in summer 2018.

\begin{table}[t] \centering
\ra{1.3}
\caption{Different qualities of the Deeplearning frameworks: PyTorch and Tensorflow)}
\begin{tabular}{@{}rrr@{}}
\toprule & PyTorch & TensorFlow \\
\midrule
Open-source									& + & + \\
Dynamic Computational Graph			& + & -  \\
Static Computational Graph				& - & +  \\
Easy Learning Curve							& + & -  \\
Fast developing of new Models			& + & -  \\
Production Environment					& - & + \\
Developer Community						& + & + \\
Native Visualization							& - & +  \\
Debugging										& + & -  \\
Data-Parallelisme								& + & -  \\
Framework-Feeling							& + & -  \\
Library-Aggregation							& - & +  \\

\bottomrule
\end{tabular}
\label{tbl:DeepLearningFrameworks}
\end{table}


For me the most important thing was the ability to work in a very pythonic way and be able to start developing my models quickly without having to learn new frameworks. Debugging is one of the most important things when learning how to code new problems and therefore my choice fell towards PyTorch.

There are some higher level frameworks like Keras and DeepDIVA that enable many more things and faster development. Keras is build on top of TensorFlow and has many models pre-implemented. It enables the developer to very quickly start modeling a problem or apply already present architectures to new data. It does not allow the same flexibility as TensorFlow but if a new architectre needs to be designed from scratch it is done in TensorFlow, than made available on Keras as to use on different datasets or different tasks. A possible counterpart for Keras is DeepDIVA that was built on top of PyTorch and also provides pre-implemented architectures or allows to include them in a straigth-forward manner. It tackles some of the disadvantages of PyTorch versus Tensorflow like including TensorBoard Visualization to PyTorch. Development in DeepDIVA is also very pythonic and does not actually change at all since it is very tightly integrated in the PyTorch framework.



Keras is a higher level framework, that comes with 

