%
% University of Fribourg
%
% Master Thesis Template
% Thomas Kolonko 13.04.2019
%

\documentclass[11pt,a4paper,twoside,hidelinks,openright]{rvsmaster}

\usepackage[latin1]{inputenc}
\usepackage{graphicx}
\usepackage{float}
\graphicspath{ {./images/} }
%%%%%%%%%%%%%%%%%%%%%%%%%%%%%%%%%%%%%%%%%%%%

\usepackage[a4paper,twoside]{geometry}

\sloppy

%% Additional packages
%%%%%%%%%%%%%%%%%%%%%%%%%%%%%%%%%%%%%%%


\usepackage{mathpazo}

% for source code highlighting
% \usepackage{listings}
% \lstloadlanguages{tcl, Perl}
% \lstset{language=tcl, commentstyle=\it, basicstyle=\tiny, keywordstyle=\bf, breaklines=true, frame=single}

% multiple figures with same general caption. FIGTOPCAP puts the subcaption on top of the figures
\usepackage[FIGTOPCAP]{subfigure}
\setcounter{lofdepth}{2}  

% offers more possibilities in captions
\usepackage{caption}


% Tool for handling the figure on multicolumns or multirows
\usepackage{multicol}
\usepackage{multirow}

% Main page with logos...
\usepackage{unifrmr}


% Fancy tables
\usepackage{booktabs}
\newcommand{\ra}[1]{\renewcommand{\arraystretch}{#1}}
\usepackage[font={small,it}]{caption} 
\captionsetup[table]{skip=10pt}
\usepackage{array}

\usepackage{listings}

\newcolumntype{L}[1]{>{\raggedright\arraybackslash}p{#1}}

% Commands for package caption
% - caption
\renewcommand{\captionfont}{\small}
\renewcommand{\captionlabelfont}{\bfseries}

% offers rotation of figures, ...
\usepackage{rotating}

% to support correct hyphenaten, add words with -
% \hyphenation{test-case}

% If a page has no content, make it an empty page (without page numbers ....)
\makeatletter
\def\cleardoublepage{\clearpage\if@twoside \ifodd\c@page\else
	\hbox{}
	\thispagestyle{empty}
	\newpage
\fi\fi}
\makeatother


% check whether we are running pdflatex
\newif\ifpdf
\ifx\pdfoutput\undefined
\pdffalse % we are not running pdflatex
\DeclareGraphicsExtensions{.eps}
\else
\pdfoutput=1 % we are running pdflatex
\pdfcompresslevel=9     % compression level for text and image;
\pdftrue
\DeclareGraphicsExtensions{.pdf,.png,.jpg}
\fi
\ifpdf
%to make table of contents and index appear in bookmarks
\usepackage{tocbibind}
%refs also as links
\usepackage[pdftex,plainpages=false]{hyperref}
%plainpages=false: enable links although page numbering is reset after title
%backref, pagebackref]{hyperref}
%\usepackage[pdftex]{color}
%\usepackage[pdftex]{thumbpdf}
\else
%url must be escaped. (this works fine in dvi)
\usepackage{url}
\fi


\newcommand{\thesistitle}{Evaluation of Deep Learning Methods on SEM Images of Asbestos}
\newcommand{\thesisauthor}{Thomas Kolonko\footnote{thomas.kolonko@students.unibe.ch, HumanTech group, University of Fribourg}}
\newcommand{\thesisleiter}{Professor Elena Mugellini, HumanTech Institute, University of Applied Sciences of Western Switzerland, Supervisor}
\newcommand{\thesisexpert}{Professor Denis Lalanne, Human-IST Institute, University Fribourg, Supervisor}
\newcommand{\thesisasst}{Jacky Casas, HumanTech Institute, University of Applied Sciences of Western Switzerland, Supervisor}
\newcommand{\thesisdate}{April, 2019}

\begin{document}


% T I T L E
% % % % % % % % % % % % % % % % % % % % % % % % % % % % % % % % % %
\begin{titlepage}  
  \begin{center}  
  
  \begin{figure}[t]  
  \vspace*{-2cm}        % to move header logo at the top 
  \center{\includegraphics[scale=0.2]{logos/MSc_quer.png}}
  \vspace{0.4in}     
  \end{figure}

    \thispagestyle{empty}
    
    {\bfseries\Huge \thesistitle \par}

    \vspace{0.3in} 
    \LARGE{\textbf{Master Thesis} \\}
    \vspace{0.4in}

    {\Large \thesisauthor}
    
    \vspace{0.3in}
    {\Large Faculty of Science - University of Bern \par}
%    {\Large Philosophisch-naturwissenschaftlichen Fakult\"{a}t \\
%            der Universit\"{a}t Bern \par}
    \vfill
    {\Large \thesisdate \par}
    \vspace{0.3in}
    {\normalsize \thesisleiter \par}
	\vspace{0.15in}    
    {\normalsize \thesisexpert \par}
    \vspace{0.15in}    
    {\normalsize \thesisasst \par}
  

  \vspace{0.9in}
 
  % === Logos ==============================================     
  \begin{figure}[htp]
    \centering
    \includegraphics[scale=0.26]{logos/UNI_Bern.png}\hfill
    \includegraphics[scale=0.28]{logos/UNI_Neuenburg.png}\hfill
    \includegraphics[scale=0.80]{logos/UNI_Fribourg.png}\hfill
    \includegraphics[scale=0.90]{logos/HumanTech.png}
  \end{figure}
  % === // Logos ===========================================    


  \end{center}

\end{titlepage}


\thispagestyle{empty}
\mbox{}

\newpage

\chapter*{Abstract}

Asbestos is a highly toxic silicate mineral that has been used widely in many products for its insulating, non-flammable and heat resistant properties. It may lead to chronic inflammation of the lungs and cancer when exposed to high concentrations. After the toxicity has become known, much effort was undertaken - and is up to this day - in removing asbestos from buildings, roofs and other materials used in industry and in the public. Asbestos detection is a manual, complex and time-intensive process, that requires a lot of experience from an expert in order to have consistent and correct results. In an attempt to reduce manual labor and increase consistency in detection, machine learning models have been recently adopted to automate the detection process.

\vspace{3mm} %5mm vertical space

Convolutional neural networks are a subset of deep learning algorithms that have been shown to be very effective for computer vision tasks, outperforming humans in many areas. They have the property of automatically extracting feature mappings from the provided dataset, which encode relevant spatial information and patterns. These properties make convolutional neural networks a very good fit for the task of detecting asbestos fibers. Especially by reducing the needed knowledge about asbestos and experience acquired through years of working with asbestos which is one of the main prerequisites of a laboratory to be successful in the detection of asbestos fibers.

\vspace{3mm} %5mm vertical space

In this thesis, several state-of-the-art architectures are explained and compared against each other in regard of performance, size, and complexity. Since the provided dataset with its 2'000 images is rather small, techniques like transfer learning, data augmentation and dataset alterations are applied and evaluated extensively. Another branch of investigation contains different modifications to the current architectures in order to achieve better performance while decreasing overall complexity. Visualization techniques should allow a better understanding in what the models learn, increase user's trust and support the reasoning of the findings.


%I will evaluate if transfer learning from the ImageNet dataset can be applied in this specific cross-domain task of asbestos fiber detection. in microscopic images. Data augmentation with different cropping methods will be implemented and compared with the goal of increasing the overall accuracy of the model. Since the dataset is quite small, with only about 2'000 images, several train and valuation datasets and their distribution are evaluated. The test set is fixed in the beginning and never altered. Another branch of investigation will be alterations to the current architectures like reducing the filters and fully connected layers in order to decrease complexity and increase interpretability through visualization. Also, adding new layers in order to allow the input image to be bigger, thus reducing problems that arise with cropping and re-sizing the original image are investigated.

\vspace{3mm} %5mm vertical space

There is no clear preference for any of the investigated architectures. They all performed in a similar range with Densenet121 achieving best results with an accuracy of 86\% when trained from scratch. Transfer learning from ImageNet was beneficial in every single case leading to an improvement of roughly 7\%. Visualizations show that this is due to mid- and high-level feature mappings transferred from ImageNet. The models trained from scratch fail to create more complex mid- and high-level filters from that few images. Data augmentation and cropping methods help to reduce the problem of overfitting and slightly increase the accuracy by 2-3\%. Dataset alterations failed to consistently increase performance. So did the architecture modifications, that allowed bigger image sizes to be fed to the network. Reducing the overall number of parameters by over 99\% did not harm the performance but reduced the complexity drastically making these models much faster, easier to interpret and more deployable.
 
%I succeeded to show that transfer learning on ImageNet does indeed benefit the asbestos detection task on microscopic images for all used architectures. I show through visualizations that this is mostly due to the mid- and high-level feature mappings transferred from ImageNet. The networks trained from scratch fail to create these more complex feature mappings altogether. The increase from pre-trained weights is overall roughly 7\% in accuracy gain. Data augmentation and cropping methods help to reduce the problem of overfitting and slightly increase the accuracy by 2-3\%. Dataset alterations fail to consistently increase accuracy, so does altering the architecture to allow bigger images as input volumens, thus reducing the problem of lost information through the resizing process. I was able to reduce the overall complexity of the networks by over 99.9\% without harming performance or even increase it in some cases. This is important because it leads to simpler models, witch faster training and better deployability.


\pagenumbering{roman}
\tableofcontents{}
\listoffigures{}
\listoftables{}
% if you have some code listing, you can include a list of all your listings
% \lstlistoflistings{}



% if you want to thank somebody for the support during your studies make it here
\chapter*{Acknowledgments}

I would first like to thank my thesis advisor and Professor Dr. Elena Mugellini for the opportunity to write this master thesis in her research group. She was always very helpful in giving advice and guidance throughout the project. Her door was always open if I needed to discuss the progress of the thesis.\\


Then, I would like to thank my coach and thesis advisor Jacky Casas. He supported me through the whole process of writing this thesis, giving me invaluable inputs and steering me into the right direction when I got lost. He was always available and took his time to listen to my ideas and propose alternatives.\\


Also special thanks to Vinay Pondenkandath and Michele Alberti, who I was always able to contact for code and DeepDIVA related issues. They always took plenty of time to explain the framework in detail and give implementation advice.\\


I would also like to thank the expert, Professor Dr. Denis Lalanne, for his time and invaluable remarks on how to improve the thesis, listen to my presentations and correct the work. \\


Many thanks to the Microscan Service SA company that provided their high-quality dataset for this master thesis.\\


I am also very grateful for the resources that were generously provided by the research group of Prof. Elena Mugellini.  \\

Lastly I would like to thank Nathalie Vielle, Jacky Casas, Matthias Fuchs and Andreas Bugmann for their time and effort in acquiring the inter-rater-agreement rate for the asbestos classification task.

\cleardoublepage
\setcounter{page}{1}	% Reset page numbering to 1
\pagenumbering{arabic}	% Arabic page numbers

% Advice: split up your thesis in multiple files, i.e. one file for one section
\chapter{Introduction}

Asbestos is a very toxic mineral that comes in one of six types. The most common being chrysotile asbestos. It has long, curly fibers which can be woven into materials and has many commercial applications. It is very durable, heat and chemical resistant and has strong insulating properties. These are the reasons why asbesots fibers were often used in thousands of products before the toxicity became widely known. Exposure with asbestos through inhalation or ingestion can lead to many asbestos-related diseases like mesothelioma (a type of cancer) or asbestosis (long term inflammation and scarring of the lungs). \\

These debilitating health effects in humans after exposure to asbestos made it very necessary to have effective techniques and methods to detect and quantify asbestos in a variety of materials and the environment. There is no one superior method in asbestos detection but rather a large number of different methods that need to be adapted to the specific task at hand. \\

Convolutional Neural Networks (CNNs) have become the standard in many areas such as image recognition, classification, object detection, identifying faces and others. Although CNNs have existed already since 1988, recent progress in computational power through GPU's have made bigger neural networks possible. In 2012, Alex Krizhevsky and his group won the ImageNet Large Scale Visual Recognition Challenge (ILSVRC) with a CNN and since then research into this type of Neural Networks has increased drastically with new CNNs surpassing human performance in many fields. \\

[[ how much references do I need here already?  I think none for the introduction but then again I make many claims that could be challenged when not providing references... ]]

\section{Motivation}

A Convolutional Neural Network is a specific type of an Artifical Neural Network (ANN) which gets its name from trying to model a biological neural network. It consists of many neurons often called nodes or units that process information in a feed forward manner. A neuron receives input from other neurons very similar to brain cells that are interconnected with other brain cells, building up a huge network. The input from many different neurons is weighted and processed before forwarding the information/output to the next neuron. A network is created where many neurons are fully connected to the next layer of neurons leading up to a Multi Layer Perceptron. \\

For image classification having fully connected layers of hidden units is not very useful, since no spatial information can be retrieved from the image itself. Convolutional Neural Networks are able to scan through the image and retrieve spatial feature maps, that find characteristic patterns throughout the image. These convolutional layers are connected to form a bigger and deeper network.  \\

\section{Study Subject}

The goal of this thesis is to find a suitable architecture to recognize asbestos in microscopic images. I will frist start with AlexNet to get a good baseline, from which to improve on. The current state of the art architectures are all based on natural images as found in the ImageNet dataset from the ILSVRC. To achieve good accuracies I will need to understand how the different layers of a CNN interact with each other in order to be able to alter the width, depth and other parameters of an architecture to better suite the given asbestos dataset. The dataloader will need re-implementation since the asbestos images are big images and resizing them leads to losing the fine structure of the asbestos fibers, making it impossible to classify the images correctly. Transfer learning will be applied in order to start with already pre-trained weights. Holding some of the layers fixed and fine-tuning the last couples of layers should improve efficiency and accuracy considerably. In the end I will need to optimize all used hyper-parameters in order to achieve best possible results and run the architecture for a long period of time.

\section{Outline}

The remainder of this thesis is organized as follows. Chapter 2 explains shortly 

\chapter{Analysis}

In this chapter I will shortly describe the dataset provided for this task, go over related work, then describe the frameworks and tools that are available. In the end I draw a conclusion and decide on how to proceed.

\section{Dataset}

The provided dataset consists of about 2'000 microscopic images with and without asbestos fibers. The images come in two different dimensions and different qualities. Most of the images are 1024 by 1024 pixels and using up 1.1 MB of disk space but some are in 1024 by 768 pixels and use only around 700 kB of disk space. The smaller images were originally in TIF format which needed to be converted into PNG format for better being able to load them into pyhton objects. All images are in grey color space. [TODO: why are there three channels then in python tensor???]. In figure \ref{fig:asbestos_examples} three images with a asbestos fiber are shown whereas in figure ... three images without asbestos are shown.

\begin{figure}[h]
\centering
\subfigure{
\includegraphics[width=.3\textwidth]{images/chapter2/asbestos_one.png}
}
\subfigure{
\includegraphics[width=.3\textwidth]{images/chapter2/asbestos_two.png}
}
\subfigure{
\includegraphics[width=.3\textwidth]{images/chapter2/asbestos_three.png}
}

\caption{Three examples of images with asbestos fibers. On the left the asbestos fibers are clearly visible, in the middle it's much harder to find them. In the right image there might be none, although the image is labelled as having asbestos in it.}
\label{fig:asbestos_examples}
\end{figure}

\begin{figure}[h]
\centering
\subfigure{
\includegraphics[width=.3\textwidth]{images/chapter2/non-asbestos_one.png}
}
\subfigure{
\includegraphics[width=.3\textwidth]{images/chapter2/non-asbestos_two.png}
}
\subfigure{
\includegraphics[width=.3\textwidth]{images/chapter2/non-asbestos_three.png}
}

\caption{Three examples of images without asbestos fibers. On the left image, there is clearly no asbestos to be found. In the middle and left image it's already much more difficult to be certain that there is none.}
\label{fig:non-asbestos_examples}
\end{figure}

As one difficult example (of many) I would like to show an image that was labeled and checked by the laboratory as having no asbestos in it. Nonetheless there are several areas where asbestos like structures emerge once the image is made brighter with an photo editing tool, especially a long fiber on the left side of the image marked by two arrows. This is to show, that the labeling will most certainly have errors in it, and that some images might look like having asbestos fibers in it but actually don't and the other way around. In order to be sure, the probes are examined chemically in a second phase that is not part of the masterthesis. Therefore it is more important to have less false positives than false negatives.

\begin{figure}[h]
\centering

\subfigure{
\includegraphics[width=.4\textwidth]{images/chapter2/probably_wrong_label.png}
}
\subfigure{
\includegraphics[width=.4\textwidth]{images/chapter2/probably_wrong_label_edited.png}
}

\caption{After brightening up the image and looking at it carefully, many different asbestos like structures emerge.}
\label{fig:non-asbestos_examples}
\end{figure}

\section{Related Work}

There are many different forms of asbestos that occur in different materials and environments, therefore the detection and quantification methods are quite different as well, although most of them are based on microscopic images. After the samples have been collected, trained experts need to quantify and classify the fibers in order to assess the contamination and risk factors to humans. The reason for that is that not all asbestos fibers are equally toxic to humans. The structure needs to be manually checked and matched exactly to a set of predefined and standardized characteristics, also called templates. This process is very time intensive and costly and therefore automated detection and counting methods are being developed. The most important and widely used methods for detecting asbestos from air samples are e.g. phase contrast microscopy (PCM), transmission electron microscopy (TEM), scanning electron microscopy (SEM) as seen in Figure \ref{fig:chrysotile}, and polarized light microscopy (PLM) \cite{perry2004discussion}. These techniques are mostly adapted for soil samples although soil samples pose many new problems like standardized preparation of the samples. Air samples are much cleaner and the asbestos fibers can be counted more easily without noise interfering, thus the results are more accurate and counts are rather reproducible. With soil samples there is much more debris in the samples and getting a homogenous sample out of a bigger object that can generalize to the whole is one of the biggest problems. Figure \ref{fig:sampleprep} shows how a very simple sample preparation could look like. \\

\begin{figure}[h]
\centering

\subfigure{
\includegraphics[width=.4\textwidth]{images/chapter2/SamplePrepOne}
}
\subfigure{
\includegraphics[width=.35\textwidth]{images/chapter2/SamplePrepTwo}
}
\subfigure{
\includegraphics[width=.4\textwidth]{images/chapter2/SamplePrepThree}
}
\subfigure{
\includegraphics[width=.35\textwidth]{images/chapter2/SamplePrepFour}
}
\caption{On the upper left image the raw material is seen that needs to be screened for asbestos. In the upper right image it has been processed in a first step manually. In the left lower image a crusher is seen, that takes the pieces and processes them into fine grained powder seen in the lower right image \cite{mohammed2015}. }
\label{fig:sampleprep}
\end{figure}


There are many other asbestos detection methods and counting strategies like using certain proteins from Escherichia coli that binds strongly to chrysotile, which in turn makes chrysotile easily detectable with fluorescent microscopy \cite{kuroda2008detection}. Other current methods are Polarized light microscopy (PLM), X-ray diffraction (XRD) and Fourier Transform Infrared Spectroscopy (FTIR) \cite{campopiano2018inter}. In a current and rather big inter-laboratory study, 475 laboratories in Italy were tested if they could reliably detect asbestos fibers in several bulk materials using the above mentioned three methods of PLM, XRD and FTIR. Many laboratories (ranging from 3\% to 40\% depending on the material and asbestos fiber) were classified as unsatisfactory having made to many errors in the classification. The authors concluded that asbestos detection is a complex process that uses several different approaches depending on the material and that the  experience and skill of the analyst is very important. Without proper training and an scientific approach it is very difficult to classify asbestos accurately \cite{campopiano2018inter}. \\

There are still many other detections methodologies but they all rely on manual screening which is very time-intensive labor. Searching on Google Scholar leads to very few papers, that applied some sort of machine learning algorithms in combination with one of the mentioned microscopy.

talk about this paper cossio2018innovative which gets accuracies of 49.88 and 78.66 \%

I will try to focus on Scanning Electron Microscopy and 
In this chapter I will give a brief overview on related work on asbestos detection and focus on asbestos fiber detection from microscopic images with the help of statistical methods and deep learning. \\




Asbestos detection can be done in a variety of ways, [what ways + ref]

Asbestos can occur in many different forms and 

Looking for publications on Google Scholar for << asbestos detection method "microscope images" >>

\section{Deeplearning Frameworks}

There are two main Deeplearning Frameworks that can readily be used. Tensorflow \cite{tensorflow} which was originally developed by researchers and engineers at Google Brain and is based on Theano and PyTorch \cite{pytorch} that was built by researchers at Facebook. They are both open source and free to use, but the flavor is quite different. While TensorFlow uses static computational graphs that need to be built prior to compilation and run in their run engine, PyTorch uses dynamic computational graphs that can be more interpreted than compiled. Programming in PyTorch is much more pythonic whereas in TensorFlow the user needs first to get used to the Tensorflow way of doing things. Like building the whole computational graph in advance, using placeholders for all weights and variables, then creating a session in which the graph can be executed. Debugging in Tensorflow is more difficult since it needs at least two different debuggers to be used. One for the tensors and their values, and one for the python code itself. That makes it much less intuitive to simply debug the underlying code while keeping track of all tensors. In PyTorch the native debugger may be used for the whole codebase, including all the variables and weights. Data parallelisme is much easier to use in PyTorch since the distribution of the code and data onto all the GPU's happens automatically. Whereas in Tensorflow much more manual work and careful thought needs to be applied to achieve the same behaviour. Since PyTorch is one big framework it gives more the feeling of working with one framework that uses a very pythonic way to handle things. TensorFlow on the other hand is more like an aggregation of many libraries that work together to achieve a common goal. Although that was the case in early 2018 things are changing very fast for these frameworks. Tensorflow was much better for production environments but with PyTorch version 1.0 this advantage is closing fast \cite{pytorchOnePointZero}. Table \ref{tbl:DeepLearningFrameworks} summarizes the different qualities of  both frameworks at the start of the Masterthesis in late summer 2018. \\

\begin{table}[t] \centering
\ra{1.3}
\caption{Different qualities of the Deeplearning frameworks: PyTorch and Tensorflow)}
\begin{tabular}{@{}rrr@{}}
\toprule & PyTorch & TensorFlow \\
\midrule
Open-source									& + & + \\
Dynamic Computational Graph			& + & -  \\
Static Computational Graph				& - & +  \\
Easy Learning Curve							& + & -  \\
Fast developing of new Models			& + & -  \\
Production Environment					& - & + \\
Developer Community						& + & + \\
Native Visualization							& - & +  \\
Debugging										& + & -  \\
Data-Parallelisme								& + & -  \\
Framework-Feeling							& + & -  \\
Library-Aggregation							& - & +  \\

\bottomrule
\end{tabular}
\label{tbl:DeepLearningFrameworks}
\end{table}


There are some higher level frameworks like Keras \cite{keras} and DeepDIVA \cite{deepdiva} that enable many more things and faster development. Keras is build on top of TensorFlow and has many models pre-implemented. It enables the developer to very quickly start modeling a problem or apply already present architectures to new data. It does not allow the same flexibility as TensorFlow but if a new architectre needs to be designed from scratch it is done in TensorFlow, than made available on Keras as to use on different datasets or different tasks. A possible counterpart for Keras is DeepDIVA that was built on top of PyTorch and also provides pre-implemented architectures or allows to include them in a straigth-forward manner. It tackles some of the disadvantages of PyTorch versus Tensorflow like including TensorBoard Visualization to PyTorch. Development in DeepDIVA is also very pythonic and does not actually change at all since it is very tightly integrated in the PyTorch framework. Creating new architectures is or altering existing ones is straight forward.

\section{Tools}

\subsection{Visualization}

WIth CNN's it is very difficult to really know what they have learned. There are several examples that an architecture performed exceptionally well on the test set but later it was found out that it didn't really recognize the object itself but some other characteristics unrelated to the studied class. This is especially important when working with only a few classes. It is indeed very unsatisfactory from a scientific standpoint to use the accuracy alone to determine if an architecture is well suited for a problem or not. Even with other statistical values added, it still lacks deeper understanding. There are several methods on how to study what a model really learns and to visualize it. I will apply a visualization toolbox. Through visualizing and understanding the learning process alterations can be applied to further improve the accuracy of the model.

\subsection{SigOpt}

Gridsearch is one of the optimization methods widely applied to a model in order to find the optimal hyper-parameters for a given task. This approach is time intensive and only touches on a few predefined values of the hyper-parameters. If several hyper-parameters need to be optimized the amound of values increases exponentially and becomes unfeasable to test them all. SigOpt applies baysien DISCUSS FURTHER

\section{Conclusion}

For me the most important thing was the ability to work in a very pythonic way and be able to start developing my models quickly without having to learn new frameworks. Debugging is one of the most crucial things when learning how to code new problems and because of these two main reasons my choice fell towards PyTorch. Additionally, I decided to go with DeepDIVA on top of PyTorch. It gives me many of the advantages that applied previously only to tensorflow like simple visualization of the results and learning process through tensorboard, but it also gives me some very useful pre-implemented architectures to start from. \\

For the visualization of the different layers of the network I decided to go with Utku Ozbulak's visualization toolbox \cite{viztoolbox}. It is implemented in PyTorch and can be easily applied with some tweakings to the current models. It has many different visualization modes to chose from and is quite extensive. It also provides a heat maps for easy localization of the object.
\chapter{CNN}

Give a short intro into CNN

\section{Design overview}

What is the architecture of CNNs`?

\section{Inception}

Explain Inception and why it's amazing

\section{ResNet}

Explain Resnet and why it's amazing

\chapter{Design and Implementation}

The main goal of this thesis is to achieve a high accuracy on the recognition of asbestos fibers in microscopic images. Since no baseline is available, one needs to be established first. The dataset consists of only a few hundred images, so data augmentation will is an important tool. Several architectures are evaluated with and without transfer-learning. Architectures are tweaked to better accomodate the ressource constraints posed by Hardware restrictions.

\section{Problem Description}

Talk about the asbestos problem. The size of the images and the GPU restrictions on memory

\section{Data Augmentation}

How and why has the data been augmented?

\section{AlexNext}

It has not been possible to get a baseline for the current performance or even an accurate estimation of human performance in the detection of asbestos fibers in the provided images. Therefore, AlexNet was used to create a baseline on which improvements may be observed and would allow discussions on architectures and their performance. The hyper-parameters learning rate and learning rate decay have been optimized with grid search by running every configuration five times, averaging the results and choosing the hyper-parameters that performed best.

\begin{table}[t] \centering
\ra{1.3}
\caption{AlexNet accuracies for baseline with optimized hyper-parameters}
\begin{tabular}{@{}rrrr@{}}
\toprule & learning rate & lr-decay & accuracy \\
\midrule
AlexNet		& 0.1 		& 5		& 52.18\%  \\
AlexNet		& 0.05 		& 5		& 53.64\%  \\
AlexNet		& 0.01 		& 5		& 53.37\%  \\
AlexNet		& 0.005 		& 5		& 53.67\%  \\
AlexNet		& 0.001 		& 5		& 78.82\%  \\
AlexNet		& 0.0005 		& 5		& 78.55\%  \\
AlexNet		& 0.0001 		& 5		& 77.09\%  \\
\bottomrule
\end{tabular}
\label{tbl:similarity-test-map}
\end{table}


\section{Inception / ResNet}

Talk about other architectures and why they are important

\section{SigOpt Optimization}

What is SigOpt and why did I use it?

\chapter{Evaluation}

As mentioned in Chapter 4 there is no reliable baseline and one needs to be established first with some basic CNN architectures. Additionally an annotators inter-agreement rate is computed to have a good guess what is realistically possible and how difficult the test set is. Then different architectures will be compared against each other with and without pre-training in order to find out if transfer learning is a valid option for this problem task. \\

The learning rate decay is defined as reducing the learning rate by a factor of 10 every N epochs and holds true for all the experiments: \\

\[ lr = lr * (0.1^{\frac{epoch}{decay}}) \] \\

For the baseline, I used a grid search approach in optimizing the learning rate and the learning rate decay. For all later architectures, I used SigOpt which utilizes bayesian optimization. With SIgOpt I followed the official guideline to run the model for 10 times for every hyperparameter which needs to be optimized. For Alexnet and all state-of-the-art architectures I optimize the following three hyperparameters: \\

\begin{itemize}
  \item learning rate [0.0001 - 0.1]
  \item momentum [0 - 1]
  \item weight\_decay [0.00001 - 0.01]
\end{itemize} 

\quad

Preliminary runs have shown that 50 epochs are long enough to converge as will be shown for every architecture separately. Running for longer did potentially further increase the accuracy of the training set but not of the test set. Sometimes it even degraded slightly due to overfitting to the training data. SigOpt optimization were all conducted with 50 epochs, so were the follow up runs.

The remaining of this chapter is in the following format. I introduce the optimized hyperparameters for each architecture and show the achieved accuracies. I compare the results with transfer learning and discuss the graphs and confusion matrices. For transfer learning a separate optimization with Sigopt is performed, since the hyperparameters need to be different if the starting point of the model is not randomly defined. Then I will attempt to shed some light into what exactly was learned by showing some of the visualizations.








\section{Annotator Inter-Agreement Rate}

The inter-agreement agreement rate gives a score of how much consensus there is between the different raters. Although it cannot be used as a baseline directly since the computation differs from the computation on how the accuracy is computed, it can show how difficult it is to classify the given test samples. For this task 4 raters were recruited a PhD candidate in Computer Science, a PhD candidate in Virology and Immunology, a PhD candidate in History and a Computer Scientist. They were given 168 images to classify into containing asbestos or not. In order to know what to look for, they were given the same (but slightly reduced in number) pre-classified images that were provided by the laboratory with a note explaining the task. They were allowed to switch forward and back between ground truth examples and the unlabeled samples. They were allowed to ask, make a break and take as much time as they saw fit in order to be certain, that they made the best possible decision.\\


\begin{table}[h] \centering
\ra{1.3}
\caption{Inter-agreement rate (Randolph's kappa) of 4 annotators.}
\resizebox{0.8\textwidth}{!}{%
\begin{tabular}{@{}rccc@{}}
\toprule & Overall Agreement & Free-margin kappa & 95\% confidence interval \\
\midrule
Randolph's kappa     & 78.27\%   & 0.57 &  [0.48, 0.65]     \\
\bottomrule
\end{tabular}}
\label{tbl:interagreement}
\end{table}

\quad

Although the overall agreement percentage is at 78.27\% the Randolph's kappa is only at 0.57. The kappa value ranges from -1 (total disagreement below chance) to 1 (total agreement above chance). Kappa values below 0.4 are considered as "poor", values from 0.4 to 0.75  are considered "intermediate to good" and values  above 0.75 are considered "excellent". Therefore the obtained value of 0.57 is between "intermediate" and "good" gives a quality measure of the reached 78.27\%. \\











\section{CNN\_Basic - Baseline}

For the baseline, a simple 3-layered CNN was chosen with three convolutions and no maxpooling. For the activation function, a leaky ReLU is applied after each convolution. A simple grid search is applied to the model as seen in Table \ref{tbl:cnn-basic-baseline}. The best result is achieved with a rather small learning rate of 0.0005 and an lr-decay of 20.\\

\begin{table*}[h]
    \ra{1.3}
    \caption{Accuracy (\%) for several learning rates and lr-decays for CNN\_Basic as a baseline.}
    \centering
    \begin{small}
    \textsc{
      \resizebox{0.99\textwidth}{!}{%
      \begin{tabular}{rcclcclcc}
      \toprule 
      & \multicolumn{2}{c}{Learning-Rate decay: 10} && %
        \multicolumn{2}{c}{Learning-Rate decay: 15} && %
        \multicolumn{2}{c}{Learning-Rate decay: 20} \\
      \cmidrule{2-3} \cmidrule{5-6}  \cmidrule{8-9}
      & learning rate & accuracy  && %
        learning rate & accuracy  && %
        learning rate & accuracy  \\ 
      \midrule
      CNN\_BASIC        & 0.1 & 69.10\%  &&  0.1 & 65.78\% &&  0.1 & 64.78\% \\
      CNN\_BASIC        & 0.05 & 63.46\%  &&  0.05 & 67.44\% && 0.05 & 65.78\% \\
      CNN\_BASIC        & 0.01 & 69.77\%  &&  0.01 & 70.10\% &&  0.01 & 69.77\% \\
      CNN\_BASIC        & 0.005 & 68.44\%  &&  0.005 & 71.43\% &&  0.005 & 72.09\% \\
      CNN\_BASIC        & 0.001 & 72.43\%  &&  0.001 & 70.10\% &&  0.001 & 74.42\% \\
      CNN\_BASIC        & 0.0005 & 73.75\%  &&  0.0005 & 73.42\% &&  \textbf{0.0005} & \textbf{76.74\%} \\
      CNN\_BASIC        & 0.0001 & 68.11\%  &&  0.0001 & 70.76\% &&  0.0001 & 70.43\% \\
    \bottomrule
    \end{tabular}}
    }
    \end{small}
    %\end{center}
    \vspace{-3.9mm}
    \label{tbl:cnn-basic-baseline}
\end{table*}

\quad


These results are much better than expected and should not be that high since the image size is resized to 32x32 pixels. As seen in Figure \ref{fig:cnn-basic} the training accuracy achieves 94.54\% while the validation accuracy achieves 71.42\%. Such a big difference between training accuracy and the final test accuracy of 76.74\% means that the model overfitted strongly to the training data, learning it too well and thus decreasing generalizability. Having the accuracies for validation and test sets near each other is good and means that the validation set is representable of the test set.

\begin{figure}[h]
\centering
\subfigure{
\includegraphics[width=.48\textwidth]{images/chapter5/TrainAccuracy.png}
}
\subfigure{
\includegraphics[width=.48\textwidth]{images/chapter5/ValAccuracy.png}
}
\caption{Training and Validation accuarcies for CNN\_basic. The vertical gray lines show when the learning rate decay happens. Convergence is reached after the 30th epoch.}
\label{fig:cnn-basic}
\end{figure}


Of course, interpreting this model is very difficult due to it's resizing of the image so early. It could be possible that the model simply learns to label all the images as non-asbestos since the dataset is strongly unbalanced but looking at the confusion matrix in Figure \ref{fig:cnn-basic-cm} shows that the model does not cheat in this way and produces sound looking results.

\begin{figure}[h]
\centering
\subfigure{
\includegraphics[width=.46\textwidth]{images/chapter5/cnn-basic-cm.png}
}
\caption{Confusion matrix of the CNN\_basic shows that the learning seems to be correct with images seperated into the two classes of asbestos and non-asbestos.}
\label{fig:cnn-basic-cm}
\end{figure}


From the confusion matrix, we can deduce type I and type II errors. Type I errors are false positives which is equivalent to a false alarm. In case of asbestos, it is better to classify wrongly some structures as asbestos than missing them which would be a type II error or a miss. Recall shows how many positive samples were actually classified as positive which in our case should be as near to 100\% as possible because it brings the type II error in relation to the overall positive samples. Recall is at 80.17\%. Precision on the other hand, how many of the predicted positive sample indeed are positive. If that number is lower, it is less severe because it simply means that more samples were classified as containing asbestos than really is the case (false alarm). Precision is at 67.83\%








%%%%%%%%%%%%%%%%%%%%%%%%%%%%%%%%%%%%%%%%%%%
%%%%%%%%%%%%%%%    ALEXNET    %%%%%%%%%%%%%%%%%%%%
%%%%%%%%%%%%%%%%%%%%%%%%%%%%%%%%%%%%%%%%%%%






\subsection{AlexNet}

AlexNet was optimized with grid search and additionally with SigOpt. With grid search the best accuracy is 81.06\% as seen in table \ref{tbl:alexnet-baseline}. It was achieved with a learning rate of 0.0001 and a learning rate decay of 20. Interestingly the SigOpt also achieved the same accuracy with a learning rate of 0.038984, a momentum of 0.073145 and a weight decay of 0.004074 as seen in table \ref{tbl:alexnet-baseline}.


\begin{table*}[h]
    \ra{1.3}
    \caption{Accuracy (\%) for several learning rates and lr-decays for AlexNet as a baseline.}
    \centering
    \begin{small}
    \textsc{
      \resizebox{0.99\textwidth}{!}{%
      \begin{tabular}{rcclcclcc}
      \toprule 
      & \multicolumn{2}{c}{Learning-Rate decay: 10} && %
        \multicolumn{2}{c}{Learning-Rate decay: 15} && %
        \multicolumn{2}{c}{Learning-Rate decay: 20} \\
      \cmidrule{2-3} \cmidrule{5-6}  \cmidrule{8-9}
      & learning rate & accuracy  && %
        learning rate & accuracy  && %
        learning rate & accuracy  \\ 
      \midrule
      AlexNet        & 0.1 & 59.80\%  &&  0.1 & 59.80\% &&  0.1 & 60.47\% \\
      AlexNet        & 0.05 & 59.80\%  &&  0.05 & 59.80\% && 0.05 & 59.80\% \\
      AlexNet        & 0.01 & 59.80\%  &&  0.01 & 59.80\% &&  0.01 & 59.80\% \\
      AlexNet        & 0.005 & 59.80\%  &&  0.005 & 59.80\% &&  0.005 & 59.80\% \\
      AlexNet        & 0.001 & 74.75\%  &&  0.001 & 77.74\% &&  0.001 & 78.74\% \\
      AlexNet        & 0.0005 & 78.74\%  &&  0.0005 & 80.40\% &&  0.0005 & 76.74\% \\
      AlexNet        & 0.0001 & 74.75\%  &&  0.0001 & 80.07\% &&  \textbf{0.0001} & \textbf{81.06\%} \\
    \bottomrule
    \end{tabular}}
    }
    \end{small}
    %\end{center}
    \vspace{-3.9mm}
    \label{tbl:alexnet-baseline}
\end{table*}

\subsection{Transfer learning on AlexNet}

When using weights from pre-training on ImageNet, the overall accuracy does get slightly better, although the increase of 0.6645\% could be due to chance. In order to show if pre-training really helps with AlexNet, each model has been run 5 times and the results are averaged as seen in Table \ref. A t-test has been run to show it's significance [NEEDS TO BE DONE]

\begin{table}[h] \centering
\ra{1.3}
\caption{Hyper parameters for Alexnet optimized with SigOpt and without pre-training}
\resizebox{0.99\textwidth}{!}{%
\begin{tabular}{@{}rrrrrrr@{}}
\toprule & learning rate & momentum & weight\_decay & lr-decay & accuracy & $\Delta$ \\
\midrule
AlexNet     from scratch    & 0.038498 & 0.073146 &  0.004074 & 20 & 81.0631\%  &         \\
AlexNet     pre-trained    & 0.034288 & 0.409378 &  0.005959 & 20 & 81.7276\%  & + 0.6645\\
\bottomrule
\end{tabular}}
\label{tbl:AlexNetBaseline}
\end{table}

\begin{table}[h] \centering
\ra{1.3}
\caption{Hyper parameters for Alexnet optimized with SigOpt and without pre-training}
\resizebox{0.99\textwidth}{!}{%
\begin{tabular}{@{}rrrrrrr@{}}
\toprule & learning rate & momentum & weight\_decay & lr-decay & accuracy & $\Delta$ \\
\midrule
AlexNet     from scratch    & 0.038498 & 0.073146 &  0.004074 & 20 & 78.61$\pm$1.16  & -     \\
AlexNet     pre-trained    & 0.034288 & 0.409378 &  0.005959 & 20 & 79.67$\pm$1.08  & - \\
AlexNet from scratch (Adam Optimizer) & 0.038498 & Adam & Adam & 20 & 59.80$\pm$0.0 & \\
\bottomrule
\end{tabular}}
\label{tbl:AlexNetMultiRun}
\end{table}

In Figure \ref{fig:alexnet-graph} we see the training and validations accuracies plotted over epochs. It can be observed that convergence is not faster with pre-training than without pre-training. Both final accuracies are almost identical. The gray vertical lines separate the graph into segments with different learning rates. So the learning rate decay happens a first time at epoch 20 and a second time at epoch 40. Convergence can be easily observed when the validation accuracy does not change anymore. Another important thing to notice is that validation and training accuracies are quite near to each other. This means that during training no severe overfitting occurred and that the validation set and train set are indeed very representative of each other. Since the final accuracy is also very near to these values it is save to say that validation and test sets are good representations of the test set and that generalizability has low variance [IF GENERALIZABILITY CAN HAVE A VARIANCE AT ALL... TODO: READ IT UP].

\begin{figure}[h]
\centering
\subfigure{
\includegraphics[width=.46\textwidth]{images/chapter5/TL/AlexNet/TA-AlexNet.png}
}
\subfigure{
\includegraphics[width=.46\textwidth]{images/chapter5/TL/AlexNet/VA-AlexNet.png}
}
\caption{No clear difference can be observed between the pre-trained AlexNet and the not pre-trained AlexNet. Both converge roughly equally and reach almost the same final accuracy.}
\label{fig:alexnet-graph}
\end{figure}

In Figure \ref{fig:alexnet-cm} the confusion matrix of the final test set is shown for both models (pre-trainend and non pre-trained). 

\begin{figure}[h]
\centering
\subfigure{
\includegraphics[width=.46\textwidth]{images/chapter5/TL/AlexNet/cm-alexnet.png}
}
\subfigure{
\includegraphics[width=.46\textwidth]{images/chapter5/TL/AlexNet/cm-alexnet-pre.png}
}
\caption{Confusion matrix from the model trained from scratch on the left side, and the model with pre-trained weights on the right.}
\label{fig:alexnet-cm}
\end{figure}








%%%%%%%%%%%%%%%%%%%%%%%%%%%%%%%%%%%%%%%%%%%
%%%%%%%%%%%%%%%%    VGG    %%%%%%%%%%%%%%%%%%%%%
%%%%%%%%%%%%%%%%%%%%%%%%%%%%%%%%%%%%%%%%%%%








\section{VGG}

For the following VGG architectures, the hyperparameter optimization was done with SigOpt. Every architecture was performed with and without batch normalization as well as with and without pre-training. Table \ref{tbl:VGG_overview} summarizes the results for the VGG architectures with the common depths of 11, 13 and 16. These are all the best achieved values in one run only.

\begin{table}[H] \centering
\ra{1.3}
\caption{Hyper parameters for VGG11, VGG13 and VGG16 with and without batch normalization (bn) optimized with SigOpt. First row of each grouping shows training the architecture from scratch. Second row of each grouping shows training with pre-trained weights from ImageNet}
\resizebox{0.99\textwidth}{!}{%
\begin{tabular}{@{}rrrrrrr@{}}
\toprule & learning rate & momentum & weight\_decay & lr-decay & accuracy & $\Delta$ \\
\midrule
VGG11 from scratch		& 0.026981 & 0.743190 & 0.01 & 20 & 80.7309\%  &         \\
VGG11 pre-trained		& 0.006787 & 0.696714 & 0.01 & 20 & 88.0399\%  & +7.309 \\
\midrule
VGG11\_bn  from scratch	&     0.059343 & 0.120718 &  0.009362 & 20 & 81.0631\%  &         \\
\textbf{VGG11\_bn pre-trained}	& \textbf{0.022571} & \textbf{0.383537} &  \textbf{0.001733} & \textbf{20} &  \textbf{89.7010\%} & \textbf{+8.6379} \\
\midrule
VGG13 from scratch		& 0.033844 & 0.257538 &  0.01 & 20 & 81.0631\%  &         \\
VGG13 pre-trained    	& 0.020012 & 0.193932 & 0.01 & 20 & 88.7043\%  & +7.6412 \\
\midrule
VGG13\_bn from scratch    			& 0.054173 & 0.643504 &  0.003223 & 20 & 81.7276\%  &         \\
\textbf{VGG13\_bn  pre-trained}    &  \textbf{0.093533} & \textbf{0.041074} &  \textbf{0.009734} & \textbf{20} & \textbf{90.3655\%}  & \textbf{+8.6379} \\
\midrule
VGG16 from scratch   	& 0.019590 & 0.383297 &  0.00001 & 20 & 81.7276\%  &         \\
\textbf{VGG16 pre-trained}    	& \textbf{0.009735} & \textbf{0.529937} & \textbf{0.01} & \textbf{20} & \textbf{90.6977\%}  & \textbf{+8.9701} \\
\midrule
VGG16\_bn from scratch	&     0.031702 & 0.370868 &  0.006288 & 20 & 83.0565\%  &         \\
VGG16\_bn pre-trained		&     0.083995 & 0.242109 &  0.002634 & 20 & 89.0366\%  & +5.9801 \\
\bottomrule
\end{tabular}}
\label{tbl:VGG_overview}
\end{table}

It can be seen that transfer learning helps in every single case and yields better performance by roughly additional 8\%. Also VGG performs generally better with batch normalization with one exception for VGG16 with pre-trained weights. This architecture outperformed its batch norm counterpart by 1.66\% which could be purely by chance. VGG13 with batch normalization seems to perform very well, although by a very small margin, since it is also less complex than VGG16, it will be used for further investigation. First, it is run three times with the found hyper parameters and then averaged as seen in Table \ref{tbl:VGG_averaged}. Since the values mentioned in Table \ref{tbl:VGG_overview} are the best values from 30 runs, it does not surprise that they are higher than the averaged values from Table \ref{tbl:VGG_averaged}.

\begin{table}[H] \centering
\ra{1.3}
\caption{Running the best VGG architecture three times with the found hyperparameters and averaging across the total of runs. }
\resizebox{0.99\textwidth}{!}{%
\begin{tabular}{@{}rrrrrrr@{}}
\toprule & learning rate & momentum & weight\_decay & lr-decay & accuracy & $\Delta$ \\
\midrule
VGG13\_bn from scratch	& 0.054173 	& 0.643504 	& 0.003223 	& 20 	& 81.0631\% $\pm$ 0.5425 &         \\
VGG13\_bn pre-trained		& 0.093533 	& 0.041074 	& 0.009734 	& 20 	& 88.0399\% $\pm$ 0.7176 & +6.9768\\
\bottomrule
\end{tabular}}
\label{tbl:VGG_averaged}
\end{table}

Looking at the training and validation graphs in Figure \ref{fig:vgg13-graph} a very similar behavior can be observed as in the previous graphs of the AlexNet architectures. Training accuracy is overfitting with the pre-trained model very quickly, convergence happens within the first 20 epochs, while training from scratch needs roughly 30 epochs to converge and never catches up with the pre-trained model. For the validation graph, convergence happens for both models shortly after 20 epochs and the validation accuracy is better with pre-trained weights (87.98\%) than from scratch (79.79\%). Both values are very near the actual values obtained in the test set which again confirms the hypothesis that the validation set does represent the test set very well.

\begin{figure}[H]
\centering
\caption{Training and validation graphs for VGG13 once with pre-trained weights from ImageNet (orange) and once trained from scratch (green)}
\subfigure{
\includegraphics[width=.46\textwidth]{images/chapter5/TL/VGG13/TA-vgg13.png}
}
\subfigure{
\includegraphics[width=.46\textwidth]{images/chapter5/TL/VGG13/VA-vgg13.png}
}
\label{fig:vgg13-graph}
\end{figure}

\subsection{VGG visualization}

In order to shed some light into what is going on, visualizing the CNN layers can show what kind of input images activate a specific filter and thus lead to a certain classification. Several hundred visualizations were checked at different layers for VGG13\_bn with and without transfer learning. In Figure \ref{fig:vgg13_pretrained_filters} 16 interesting layer activations from the pretrained model are shown. Filters that resembled rocks, background or fiber-like structures were selected from 4 different layers.

\begin{figure}[H]
\centering
\caption{Each line represents one convolution layer in the VGG13\_bn pre-trained architecture in ascending order. Four interesting activations are shown per layer starting at layer 2, layer 4, layer 7 and layer 9.}
\subfigure{
\includegraphics[width=.23\textwidth]{images/chapter5/vgg13-bn-pre/l3-f0.jpg}
}
\subfigure{
\includegraphics[width=.23\textwidth]{images/chapter5/vgg13-bn-pre/l3-f1.jpg}
}
\subfigure{
\includegraphics[width=.23\textwidth]{images/chapter5/vgg13-bn-pre/l3-f2.jpg}
}
\subfigure{
\includegraphics[width=.23\textwidth]{images/chapter5/vgg13-bn-pre/l3-f3.jpg}
}

\subfigure{
\includegraphics[width=.23\textwidth]{images/chapter5/vgg13-bn-pre/l10-f0.jpg}
}
\subfigure{
\includegraphics[width=.23\textwidth]{images/chapter5/vgg13-bn-pre/l10-f1.jpg}
}
\subfigure{
\includegraphics[width=.23\textwidth]{images/chapter5/vgg13-bn-pre/l10-f2.jpg}
}
\subfigure{
\includegraphics[width=.23\textwidth]{images/chapter5/vgg13-bn-pre/l10-f3.jpg}
}

\subfigure{
\includegraphics[width=.23\textwidth]{images/chapter5/vgg13-bn-pre/l21-f0.jpg}
}
\subfigure{
\includegraphics[width=.23\textwidth]{images/chapter5/vgg13-bn-pre/l21-f1.jpg}
}
\subfigure{
\includegraphics[width=.23\textwidth]{images/chapter5/vgg13-bn-pre/l21-f2.jpg}
}
\subfigure{
\includegraphics[width=.23\textwidth]{images/chapter5/vgg13-bn-pre/l21-f3.jpg}
}

\subfigure{
\includegraphics[width=.23\textwidth]{images/chapter5/vgg13-bn-pre/l28-f0.jpg}
}
\subfigure{
\includegraphics[width=.23\textwidth]{images/chapter5/vgg13-bn-pre/l28-f1.jpg}
}
\subfigure{
\includegraphics[width=.23\textwidth]{images/chapter5/vgg13-bn-pre/l28-f2.jpg}
}
\subfigure{
\includegraphics[width=.23\textwidth]{images/chapter5/vgg13-bn-pre/l28-f3.jpg}
}
\label{fig:vgg13_pretrained_filters}
\end{figure}

It can be very well observed that with increasing depth of the network more complex patterns are learned. In Figure \ref{fig:vgg13_app_pretrained_filters} in the Appendix, four images are shown for every layer in the architecture. The first layer learns only to spot plain colors, but already the second layer starts to look for vertical and horizontal rock-like patterns, as seen in Figure \ref{fig:vgg13_fromscratch_filters}. Some of the filters don't resemble anything from the training set but still have a high degree of pattern. These patterns were transferred over from ImageNet and adapted through 50 epochs of learning. Figure \ref{fig:vgg13_fromscratch_filters} shows four layers of the pre-trained VGG13 architectue.

\begin{figure}[H]
\centering
\caption{Each line represents one convolution layer in the VGG13\_bn architecture which was trained from scratch in ascending order. Four interesting activations are shown per layer starting at layer 2, layer 4, layer 7 and layer 9.}
\subfigure{
\includegraphics[width=.23\textwidth]{images/chapter5/vgg13-bn/l3-f0.jpg}
}
\subfigure{
\includegraphics[width=.23\textwidth]{images/chapter5/vgg13-bn/l3-f1.jpg}
}
\subfigure{
\includegraphics[width=.23\textwidth]{images/chapter5/vgg13-bn/l3-f2.jpg}
}
\subfigure{
\includegraphics[width=.23\textwidth]{images/chapter5/vgg13-bn/l3-f3.jpg}
}

\subfigure{
\includegraphics[width=.23\textwidth]{images/chapter5/vgg13-bn/l10-f0.jpg}
}
\subfigure{
\includegraphics[width=.23\textwidth]{images/chapter5/vgg13-bn/l10-f1.jpg}
}
\subfigure{
\includegraphics[width=.23\textwidth]{images/chapter5/vgg13-bn/l10-f2.jpg}
}
\subfigure{
\includegraphics[width=.23\textwidth]{images/chapter5/vgg13-bn/l10-f3.jpg}
}

\subfigure{
\includegraphics[width=.23\textwidth]{images/chapter5/vgg13-bn/l21-f0.jpg}
}
\subfigure{
\includegraphics[width=.23\textwidth]{images/chapter5/vgg13-bn/l21-f1.jpg}
}
\subfigure{
\includegraphics[width=.23\textwidth]{images/chapter5/vgg13-bn/l21-f2.jpg}
}
\subfigure{
\includegraphics[width=.23\textwidth]{images/chapter5/vgg13-bn/l21-f3.jpg}
}

\subfigure{
\includegraphics[width=.23\textwidth]{images/chapter5/vgg13-bn/l28-f0.jpg}
}
\subfigure{
\includegraphics[width=.23\textwidth]{images/chapter5/vgg13-bn/l28-f1.jpg}
}
\subfigure{
\includegraphics[width=.23\textwidth]{images/chapter5/vgg13-bn/l28-f2.jpg}
}
\subfigure{
\includegraphics[width=.23\textwidth]{images/chapter5/vgg13-bn/l28-f3.jpg}
}
\label{fig:vgg13_fromscratch_filters}
\end{figure}

It is very interesting to observe that no complex patterns can be found in these layer activations instead, there is a lot of noise in the images. Although rock like structures could be interpreted into these visualizations, fiber like structures are missing completely.

Applying Grad-CAM and heatmap visualizations can show important regions of the image that correspond to a certain classification. These areas are colored in deep red to orange, whereas areas that did not contribute are colored in blue to violet. Figure \ref{fig:vgg13_pre_heatmap} shows two Gad-CAM visualizations from different layers. In the first row the second convolution layer is shown with the network recognizing rather fine-grained regions detected by the low-level feature maps. The second row shows the last convolution layer where high-level feature maps are captured.

\begin{figure}[H]
\centering
\caption{Grad-Cam and heatmap visualizations are shown for the pre-trained architectures for layers 3 and 31}
\subfigure{
\includegraphics[width=.25\textwidth]{images/chapter5/vgg13-bn-pre/7-Cam-Grayscale.png}
}
\subfigure{
\includegraphics[width=.25\textwidth]{images/chapter5/vgg13-bn-pre/7-Cam-Heatmap.png}
}
\subfigure{
\includegraphics[width=.25\textwidth]{images/chapter5/vgg13-bn-pre/7-Cam-Image.png}
}

\subfigure{
\includegraphics[width=.25\textwidth]{images/chapter5/vgg13-bn-pre/31-Cam-Grayscale.png}
}
\subfigure{
\includegraphics[width=.25\textwidth]{images/chapter5/vgg13-bn-pre/31-Cam-Heatmap.png}
}
\subfigure{
\includegraphics[width=.25\textwidth]{images/chapter5/vgg13-bn-pre/31-Cam-Image.png}
}
\label{fig:vgg13_pre_heatmap}
\end{figure}

In Figure \ref{fig:vgg13_heatmap} the same Grad-CAM visualizations are shown at the same layers for the model trained from scratch. It can be easily seen that although the low-level features are still similar the last convolution layer yields completely different results.

\begin{figure}[H]
\centering
\caption{Grad-Cam and heatmap visualizations are shown for the pre-trained architectures for layers 3 and 31}
\subfigure{
\includegraphics[width=.25\textwidth]{images/chapter5/vgg13-bn/7-Cam-Grayscale.png}
}
\subfigure{
\includegraphics[width=.25\textwidth]{images/chapter5/vgg13-bn/7-Cam-Heatmap.png}
}
\subfigure{
\includegraphics[width=.25\textwidth]{images/chapter5/vgg13-bn/7-Cam-Image.png}
}


\subfigure{
\includegraphics[width=.25\textwidth]{images/chapter5/vgg13-bn/31-Cam-Grayscale.png}
}
\subfigure{
\includegraphics[width=.25\textwidth]{images/chapter5/vgg13-bn/31-Cam-Heatmap.png}
}
\subfigure{
\includegraphics[width=.25\textwidth]{images/chapter5/vgg13-bn/31-Cam-Image.png}
}
\label{fig:vgg13_heatmap}
\end{figure}











%%%%%%%%%%%%%%%%%%%%%%%%%%%%%%%%%%%%%%%%%%%
%%%%%%%%%%%%%%%%%    RESNET    %%%%%%%%%%%%%%%%%%%
%%%%%%%%%%%%%%%%%%%%%%%%%%%%%%%%%%%%%%%%%%%




\section{ResNet}

ResNet18 and ResNet34 were both optimized with SigOpt only. For each hyper parameter 10 runs were conducted summing up to 30 runs. It was used for training the architecture from scratch and with pre-trained weights. Table \ref{tbl:ResNet_overview} shows a summary of the optimization process with the best accuracies and the delta between the pre-trained model and the model trained from scratch.

\begin{table}[h] \centering
\ra{1.3}
\caption{Hyper parameters for ResNet18 and ResNet34 optimized with SigOpt. First row shows hyperparameters training the architecture from scratch. Second row used pre-trained weights from ImageNet}
\resizebox{0.99\textwidth}{!}{%
\begin{tabular}{@{}rrrrrrr@{}}
\toprule & learning rate & momentum & weight\_decay & lr-decay & accuracy & $\Delta$ \\
\midrule
ResNet18     from scratch    & 0.033678 & 0.952630 &  0.007518 & 20 & 83.0565\%  &         \\
ResNet18     pre-trained    &     0.039918 & 0.170826 &  0.001980 & 20 & 88.0399\%  & + 4.9834\\
\midrule
ResNet34     from scratch    & 0.040323 & 0.141704 &  0.004044 & 20 & 83.0565\%  &         \\
ResNet34     pre-trained    &     0.060848 & 0.460187 &  0.000537 & 20 & 87.0431\%  & +3.9866\\
\bottomrule
\end{tabular}}
\label{tbl:ResNet18}
\end{table}

ResNet18 and ResNet34 perform equally well when trained form scratch but differ slightly when pre-trained weights from ImageNet are used. The increased gain from transfer learning for ResNet18 is thus higher and since it has less parameters, it is the preferred architecture to be further evaluated.\\

The graphs in Figure \ref{fig:resnet18-graph} show clearly that training converged much faster for the model with pre-trained weights. After only a few epochs it reached convergence while the model without pre-trained weights had to catch up over many epochs. Both models overfit strongly to the training data and yield lower results in the validation and test sets. Interestingly the validation accuracy of the model trained from scratch starts to decrease around epoch 30 although training accuracy still rises. This is considered bad and is also reflected in the final accuracies where the model without transfer learning scores around 5\% lower.

\begin{figure}[h]
\centering
\caption{Training and validation graphs for ResNet18 once with pre-trained weights from ImageNet and once trained from scratch.}
\subfigure{
\includegraphics[width=.46\textwidth]{images/chapter5/TL/ResNet18/TA-ResNet18.png}
}
\subfigure{
\includegraphics[width=.46\textwidth]{images/chapter5/TL/ResNet18/VA-ResNet18.png}
}
\label{fig:resnet18-graph}
\end{figure}

With the confusion matrix in Figure \ref{fig:resnet18-cm} we can see that the learning did not take any shortcuts and that the distinction between images with asbestos fibers and images without were clearly separated to some extent that is reflected in the overall accuracy reached. The Precision and Recall for the model trained from scratch are 77.78\% and 80.99\%, respectively. For the model with pre-trained weight, Precision and Recall are much better with 83.47\% and 87.60\%, respectively.

\begin{figure}[h]
\centering
\caption{On the left: confusion matrix from model trained from scratch. On the right trained with transfer learning applied.}
\subfigure{
\includegraphics[width=.46\textwidth]{images/chapter5/TL/ResNet18/cm-resnet18.png}
}
\subfigure{
\includegraphics[width=.46\textwidth]{images/chapter5/TL/ResNet18/cm-resnet18-pre.png}
}
\label{fig:resnet18-cm}
\end{figure}











%%%%%%%%%%%%%%%%%%%%%%%%%%%%%%%%%%%%%%%%%%%
%%%%%%%%%%%%%%%    DENSENET121    %%%%%%%%%%%%%%%%%
%%%%%%%%%%%%%%%%%%%%%%%%%%%%%%%%%%%%%%%%%%%



\section{Densenet121}

Densenet121 and densenet169 were both optimized with SigOpt only. For each hyper parameter 10 runs were conducted summing up to 30 runs. It was used for training the architecture from scratch and with pre-trained weights. Table \ref{tbl:Densenet121_overview} shows a summary of the optimization process with the best accuracies and the delta between the pre-trained model and the model trained from scratch.

\begin{table}[h] \centering
\ra{1.3}
\caption{Hyper parameters for densenet121 optimized with SigOpt. First row shows hyperparameters training the architecture from scratch. Second row used pre-trained weights from ImageNet}
\resizebox{0.99\textwidth}{!}{%
\begin{tabular}{@{}rrrrrrr@{}}
\toprule & learning rate & momentum & weight\_decay & lr-decay & accuracy & $\Delta$ \\
\midrule
Densenet121     from scratch    & 0.035925 & 0.057618 &  0.009241 & 20 & 86.0465\%  &         \\
Densenet121     pre-trained    &     0.018489     & 0.369998 &  0.004963 & 20 & 88.3721\%  & +2.3256\\
\midrule
Densenet169 from scratch    & 0.005812 & 0.777249 & 0.006999  & 20 & 85.3821\%  &         \\
Densenet169 pre-trained    & 0.006347 & 0.447591 & 0.005180 & 20 & 89.3688\%  & +3.9867\\
\bottomrule
\end{tabular}}
\label{tbl:Densenet121_overview}
\end{table}

As already seen in the previous comparisons of VGG and ResNet architectues, the parameter size does not seem to play a role in this specific task. If the model is trained from scratch, densenet121 achieves slightly higher accuracies than densenet161. The opposite is true for the model trained with transfer learning where densenet169 performs slightly better. Since the difference is very small densenet121 is preferred for further investigation over the deeper densenet169 architecture.

The graphs in Figure \ref{fig:densenet121-graph} show again a very similar and alread known behavior. Convergence happens much faster with the pre-trained model.

\begin{figure}[h]
\centering
\caption{Training and validation graphs for densenet121 once with pre-trained weights from ImageNet and once trained from scratch.}
\subfigure{
\includegraphics[width=.46\textwidth]{images/chapter5/TL/Densenet121/TA-densenet121.png}
}
\subfigure{
\includegraphics[width=.46\textwidth]{images/chapter5/TL/Densenet121/VA-densenet121.png}
}
\label{fig:densenet121-graph}
\end{figure}





%%%%%%%%%%%%%%%%%%%%%%%%%%%%%%%%%%%%%%%%%%%
%%%%%%%%%%%%%%%    INCEPTIONv3   %%%%%%%%%%%%%%%%%%
%%%%%%%%%%%%%%%%%%%%%%%%%%%%%%%%%%%%%%%%%%%









\section{Inception}


Hyperparameter optimization for inception v3 was done with SigOpt and for both variants, training from scratch and training with pre-trained weights from ImageNet.

\begin{table}[h] \centering
\ra{1.3}
\caption{Hyper parameters for inception v3 optimized with SigOpt. First row shows hyperparameters training the architecture from scratch. Second row used pre-trained weights from ImageNet}
\resizebox{0.99\textwidth}{!}{%
\begin{tabular}{@{}rrrrrrr@{}}
\toprule & learning rate & momentum & weight\_decay & lr-decay & accuracy & $\Delta$ \\
\midrule
Inception v3 from scratch    & 0.070046 & 0.910505 & 0.006943 & 20 & 83.3887\%  &         \\
Inception v3 pre-trained    & 0.029269 & 0.0 & 0.006320 & 20 & 88.7043\%  & +5.3156\\
\bottomrule
\end{tabular}}
\label{tbl:Inceptionv3}
\end{table}

\section{Conclusion on Transfer Learning}

Using pre-trained weights from ImagNet increased accuracy of every single run performed on the above mentioned architectures. Apparently the learned feature representations are general enough to be of high  value regardles of the target domain. These findings align perfectly well with other research done on cross-domain transfer learning. By average ALKJJDLFKJDFLKJLKDFJ

Overfitting was often a problem and this should be addressed through cropping and randomization

At first it was very surprising that learning from scratch never caught up with the models using transfer learning from ImageNet. I was certain that given enough epochs and different runs the models from scratch should find at least equally suited feature mappings with random initialization of weights. But visualizing  some of the filters quickly made it clear that learning from scratch  with the provided dataset doesn't create elaborate feature mappings as seen from ImageNet. I argue that although the vast majority of feature mappings are absolutely useless for the asbestos classification, some feature mappings of e.g. grass or hair might be similar enough for the filter to activate maximally. Since the asbestos task is only a binary classification few of these filters are enogh to predict asbestos with a high accuracy. These elaborate feature mappings are not able to be created from scratch with a dataset of only a few hundred, partially wrongly labeled images, thus the model trained from scratch never reaches the accuracy of the pre-trained model.


\chapter{Evaluation on other Techniques}

Transfer learning is only one way to improve performance with only a small amount of data. In this chapter, other techniques are explored that could prove valuable for the asbestos dataset like altering the size of the training data. Less but better quality images could lead to better performance, but also increasing the training data by adding the images from the validation set to the training set could provide a boost in performance. Cropping and resizing images can increase the dataset many fold. And last but not least, alterations to the networks will be done, in order to make the network smaller and thus more efficient or make the network able to cope with bigger input images.

\section{Evaluation of overall number of parameters in VGG13}

In this section, I want to understand how the overall number of parameters affects the performance of the network. VGG13 was built primarily for the ImageNet Challenge and is probably too big for a task like the asbestos detection. In a first step, I will try to reduce the fully connected layers since they contribute to the complexity of the model the most. I will gradually reduce the number of parameters in the last 3 layers and compare the performance by averaging over 3 runs. Table \ref{tbl:vgg13_fc} shows the results. The last fully connected layer can safely be reduced from initially having 4096 units per layer to 16 units without harming the performance, or even increase it slightly while reducing the overall parameters by 92.39\%. It can be observed that applying transfer learning on a slightly altered architecture quickly diminishes it's advantages as expected. This can be explained because the architecture was relying on a specific configuration. Changing filters or fully connected layers might lead to having to retrain the whole architecture although the drop in performance is surprising since the changes only affect the last fully connected layers of the network and leave all previous filters intact.


\begin{table}[!h] \centering
\ra{1.3}
\caption{Variations in the last 3 fully connected layers of the VGG13 architecture. They were all performed with Batch Normalization. The last column shows how much parameters remain in the varied architecture relative to the original VGG13 implementation.}
\resizebox{0.9\textwidth}{!}{%
\begin{tabular}{@{}rcccc@{}}
\toprule & Accuracy in \% (from scratch) & Accuracy in \% (pre-trained) & Number of Parameters & \% of Original \\
\midrule
VGG13\_4096    	& 81.06 $\pm$ 0.54 		& 88.04 $\pm$ 0.72 		& 128'959'042 	& 100\% \\
VGG13\_1024    	& 80.07 $\pm$ 0.72 		& 81.06 $\pm$ 0.72 		& 36'153'666 	& 28.04\% \\
VGG13\_512    		& 80.84 $\pm$ 0.78 		& 80.07 $\pm$ 0.27 		& 22'520'130 	& 17.46\% \\
VGG13\_256    		& 80.95 $\pm$ 0.87 		& 80.18 $\pm$ 0.87 		& 15'899'970 	& 12.33\% \\
VGG13\_128    		& 81.40 $\pm$ 0.98 		& 79.73 $\pm$ 1.41 		& 12'639'042 	& 9.80\% \\
VGG13\_64    		& 81.62 $\pm$ 0.41 		&  81.17 $\pm$ 0.41 		& 11'020'866 	& 8.55\% \\
VGG13\_32    		& 80.07 $\pm$ 2.36 		& 81.50 $\pm$ 0.56 		& 10'214'850 	& 7.92\% \\
\textbf{VGG13\_16}    & \textbf{82.06 $\pm$ 1.65} & \textbf{79.07 $\pm$ 0.81} & \textbf{9'812'610} & \textbf{7.61\%}\\
VGG13\_8    			& 78.74 $\pm$ 1.51 		& 80.18 $\pm$ 1.28 		& 9'611'682 		& 7.45\% \\
VGG13\_4    			& 79.73 $\pm$ 0.98 		& 80.18 $\pm$ 1.92 		& 9'511'266 		& 7.38\% \\
VGG13\_2    			& 59.80 $\pm$ 0.00 	& 74.31 $\pm$ 10.28 	& 9'461'070 		& 7.36\% \\
\bottomrule
\end{tabular}}
\label{tbl:vgg13_fc}
\end{table}

As a next step, the fully connected layer is held fixed at the original size of 4096 and the filters of all previous layers are reduced gradually. First, they are all halved until the first layer reaches 16 filters instead of the original 64. Then the layers deeper within the network are halved until all layers have 16 filters. After that configuration is reached, all filters all reduced gradually until only 2 filters per layer remain. Table \ref{tbl:vgg_filter_scheme} shows a summary of the filter reduction scheme. The last filter scheme "I" is to see if there is any benefit in increasing filters starting from 2 to the last layer. This imitates the original ratio of the filters but reducing the overall parameters drastically. \\


\begin{table}[!h] \centering
\ra{1.3}
\caption{Filter reduction scheme used on the original VGG13 architecture. The fully connected layer at the end is hold fixed and only the intermediate layers are reduced in filter size. The M stands for a max pooling layer.}
\resizebox{0.73\textwidth}{!}{%
\begin{tabular}{@{}rr@{}}
\toprule & VGG13 architecture variations in number of filters\\
\midrule
Original        & [64, 64, M, 128, 128, M, 256, 256, M, 512, 512, M, 512, 512, M]  \\
A                & [32, 32, M, 64, 64, M, 128, 128, M, 256, 256, M, 256, 256, M]  \\
B                & [16, 16, M, 32, 32, M, 64, 64, M, 128, 128, M, 128, 128, M]  \\
C                & [16, 16 M, 16, 16, M, 32, 32, M, 64, 64, M, 64, 64, M]  \\
D                & [16, 16, M, 16, 16, M, 16, 16, M, 32, 32, M, 32, 32, M]  \\
E                & [16, 16, M, 16, 16, M, 16, 16, M, 16, 16, M, 16, 16, M]  \\
F                & [8, 8, M, 8, 8, M, 8, 8, M, 8, 8, M, 8, 8, M]  \\
G                & [4, 4 M, 4, 4, M, 4, 4, M, 4, 4, M, 4, 4, M]  \\
H                & [2, 2, M, 2, 2, M, 2, 2, M, 2, 2, M, 2, 2, M]  \\
I                & [2, 2, M, 4, 4, M, 8, 8, M, 16, 16, M, 32, 32, M]  \\
J                & [2, 2, M, 4, 4, M, 8, 8, M, 16, 16, M, 32, 32, M]  \\
\bottomrule
\end{tabular}}
\label{tbl:vgg_filter_scheme}
\end{table}

The goal is to understand how the filters can be reduced and if a pattern can be observed. The results are summarized in Table \ref{tbl:vgg13_fc}. It can be shown that reducing the filters from the original schema to the scheme "A" does no harm the accuracy but reduces the trainable parameters by roughly 50\%. The standard deviation of 0.2721 gives a rather robust result and is lower than the standard deviation of 0.5425 obtained with the original scheme. Surprising is the result obtained with scheme "G" which uses only 4 filters per each layer reducing the overall parameters by 86.35\%. Although it's worse compared to the original scheme, it suffered only 1.7\% which could be attributed to chance. \\

\begin{table}[!h] \centering
\ra{1.3}
\caption{Variations in number of filters throughout the VGG13 architecture as explained in table \ref{tbl:vgg_filter_scheme}.}
\resizebox{0.9\textwidth}{!}{%
\begin{tabular}{@{}rrrrr@{}}
\toprule & Accuracy in \% (from scratch) & Accuracy in \% (pre-trained) & Number of Parameters & \% of Original \\
\midrule
VGG13\_original 	& 81.06 $\pm$ 0.54 	& 88.04 $\pm$ 0.72 	& 128'959'042 	& 100\% \\
VGG13\_a    			& 81.06 $\pm$ 0.27	& 74.64 $\pm$ 4.46 & 70'529'186 	& 54.69\% \\
VGG13\_b    			& 80.40 $\pm$ 0.27 	& 80.18 $\pm$ 1.85 	& 43'073'874 	& 33.40\% \\
VGG13\_c    			& 78.74 $\pm$ 0.98	& 80.07 $\pm$ 0.98 & 29'790'002 	& 23.10\% \\
\textbf{VGG13\_d}    			& \textbf{79.40 $\pm$ 0.81} 	& \textbf{80.40 $\pm$ 0.47} & \textbf{23'261'010} 	& \textbf{18.04\%} \\
VGG13\_e    			& 78.96 $\pm$ 1.02 	& 79.73 $\pm$ 1.90	& 20'026'514 	& 15.53\% \\
VGG13\_f    			& 78.74 $\pm$ 0.81 	& 79.96 $\pm$ 0.68 	& 18'404'874 	& 14.27\% \\
\textbf{VGG13\_g}    			& \textbf{79.29 $\pm$ 0.68} 	& \textbf{79.29 $\pm$ 1.22} 	& \textbf{17'597'942} 	& \textbf{13.65\%} \\
VGG13\_h    			& 78.18 $\pm$ 0.41	& 78.63 $\pm$ 0.56 	& 17'195'448 	& 13.33\% \\
\midrule
VGG13\_i    			& 79.18 $\pm$ 2.21 	& 79.73 $\pm$ 1.18 	& 23'234'952 	& 18.02\% \\
\bottomrule
\end{tabular}}
\label{tbl:vgg13_fc}
\end{table}

Combining both modifications leads to yet another surprise as seen in Table \ref{tbl:vgg13_f_fc}. The modification with only 16 units per fully-connected layer and scheme "D" results in 80.73\% accuracy while reducing the overall complexity of the model by 99.946\%! The time needed to train the model drops from initially 1 hour and 13 minutes to only 25 minutes. Due to the heavy parallelization of the GPU's the drop in the time taken to learn is not linear at all. But more importantly is that the size of the model is much more deployable for production, e.g. IOT devices. \\


\begin{table}[!h] \centering
\ra{1.3}
\caption{Variations in number of Filters throughout the VGG13 architecture as explained in table \ref{tbl:vgg_filter_scheme} combined with the reduction of the hidden layers in the last fully-connected layer.}
\resizebox{0.9\textwidth}{!}{%
\begin{tabular}{@{}rcccc@{}}
\toprule & Accuracy in \% (from scratch) & Run time (hh:mm:ss) & Number of Parameters & \% of Original \\
\midrule
VGG13\_original    				& 81.06 $\pm$ 0.54 & 1:13:04 		& 128'959'042 	& 100\% \\
VGG13\_16\_a            			& 80.62 $\pm$ 1.25 & 0:27:01 		& 2'556'386 		& 1.9823\% \\
VGG13\_16\_b            			& 80.62 $\pm$ 0.63 & 0:25:37 		& 690'834 		& 0.5357\% \\
\textbf{VGG13\_16\_d}       	& \textbf{80.73 $\pm$ 1.44} & \textbf{0:25:35} & \textbf{70'290} & \textbf{0.0545\%} \\
\textbf{VGG13\_16\_g}       	& \textbf{79.29 $\pm$ 1.81} & \textbf{0:25:34} & \textbf{4'982} & \textbf{0.0039\%} \\
\bottomrule
\end{tabular}}
\label{tbl:vgg13_f_fc}
\end{table}

Re-optimizing the hyperparameters with SigOpt yielded small improvements as seen in Table \ref{tbl:vgg13_f_fc_optimized}. \\


\begin{table}[!h] \centering
\ra{1.3}
\caption{SigOpt opimization VGG13 architecture as explained in table \ref{tbl:vgg_filter_scheme}.}
\resizebox{0.9\textwidth}{!}{%
\begin{tabular}{@{}rcccc@{}}
\toprule & Accuracy in \% (from scratch) & Run time (hh:mm:ss) & Number of Parameters & \% of Original \\
\midrule
VGG13\_original    		& 81.06 $\pm$ 0.54 		& 1:13:04 		& 128'959'042 	& 100\% \\
VGG13\_16\_d            	& 81.28 $\pm$ 1.10 		& 0:25:34 	& 70'290 			& 0.0545\% \\
VGG13\_16\_g            	& 80.07 $\pm$ 0.47 		& 0:25:08 	& 4'982 			& 0.0039\% \\
\bottomrule
\end{tabular}}
\label{tbl:vgg13_f_fc_optimized}
\end{table}

\subsubsection{Visualization of VGG}

First, let's visualize the architecture with the best results obtained which is the pre-trained VGG13 model with batch normalization. The Grad-CAM was taken on the last convolutional layer and can be seen in Figure \ref{fig:asbestos_gradcam}. 

\begin{figure}[!h]
\centering
\caption{Grad-CAM visualization applied on the left image once focusing on the areas that lead to the classification "non-asbestos" in the first row. And once focusing on the classification "asbestos" on the second row. Areas in red show high relevance and violet shows low relevance for the classification.}
\subfigure{
\includegraphics[width=.3\textwidth]{images/chapter6/vgg13/asbestos-original.png}
}
\subfigure{
\includegraphics[width=.3\textwidth]{images/chapter6/vgg13/0-tl31-Cam-Heatmap.png}
}
\subfigure{
\includegraphics[width=.3\textwidth]{images/chapter6/vgg13/0-tl31-Cam-On-Image.png}
}

\subfigure{
\includegraphics[width=.3\textwidth]{images/chapter6/vgg13/asbestos-original.png}
}
\subfigure{
\includegraphics[width=.3\textwidth]{images/chapter6/vgg13/1-tl31-Cam-Heatmap.png}
}
\subfigure{
\includegraphics[width=.3\textwidth]{images/chapter6/vgg13/1-tl31-Cam-On-Image.png}
}
\label{fig:asbestos_gradcam}
\end{figure}

Although the asbestos is not clearly highlighted in the Grad-CAM visualization the two rows seen in Figure \ref{fig:asbestos_gradcam} show how the network looks for different patterns when classification for either category is done. As mentioned in the conclusion in chapter 5 the network probably uses feature maps that somehow resemble asbestos fibers and uses them to classify the images, thus leading to rather unclear feature maps. Also, it is possible that taking the last layer is detrimental for the asbestos task since the asbestos structure is not something as abstract and complex as a dog's face, thus the asbestos task could rely more on the low-level features. 

Visualizing images that activate certain filters in the last layer maximally are seen in Figure \ref{fig:vgg13_filter_activation}. The images are very similar to the visualizations on ImageNet with structures that do not resemble asbestos at all. Nonetheless, they achieve the best performance so far as explained in the previous chapter. I argue that some filter from ImageNet do indeed capture another structure in real life that resembles that of asbestos fibers, and if these very few filters are activated strongly. The classification is done on these very few filters and the rest (the big majority) is ignored. That's why reducing the last fully-connected layer to 16 units does not harm the performance at all.

\begin{figure}[!h]
\centering
\caption{Last layer visualizations (8 of 512) of the pre-trained VGG13 with batch normalization.}
\subfigure{
\includegraphics[width=.23\textwidth]{images/chapter6/vgg13/layers-pretrained/l22-f1.jpg}
}
\subfigure{
\includegraphics[width=.23\textwidth]{images/chapter6/vgg13/layers-pretrained/l22-f2.jpg}
}
\subfigure{
\includegraphics[width=.23\textwidth]{images/chapter6/vgg13/layers-pretrained/l22-f3.jpg}
}
\subfigure{
\includegraphics[width=.23\textwidth]{images/chapter6/vgg13/layers-pretrained/l22-f4.jpg}
}
\subfigure{
\includegraphics[width=.23\textwidth]{images/chapter6/vgg13/layers-pretrained/l22-f5.jpg}
}
\subfigure{
\includegraphics[width=.23\textwidth]{images/chapter6/vgg13/layers-pretrained/l22-f6.jpg}
}
\subfigure{
\includegraphics[width=.23\textwidth]{images/chapter6/vgg13/layers-pretrained/l22-f7.jpg}
}
\subfigure{
\includegraphics[width=.23\textwidth]{images/chapter6/vgg13/layers-pretrained/l22-f8.jpg}
}
\label{fig:vgg13_filter_activation}
\end{figure}

In order to contrast the original and pre-trained VGG13 with the modification performed on the VGG13 architecture with scheme "G" and only 16 units in the fully connected layers, I performed the same visualizations again. In Figure \ref{fig:asbestos_gradcam_g} Grad-CAM visualizations with heatmaps over the image are shown for the last convolution layer. As before the first row shows what areas are responsible for the classification "non-asbestos", and the second row shows what areas are responsible for the classification "asbestos".

\begin{figure}[!h]
\centering
\caption{GradCam visualization applied on the left image once with looking for non-asbestos structures (first row) and once looking for asbestos structures (second row).}
\subfigure{
\includegraphics[width=.3\textwidth]{images/chapter6/vgg13/asbestos-original.png}
}
\subfigure{
\includegraphics[width=.3\textwidth]{images/chapter6/vgg13/layers-g-16/0-tl31-Cam-Heatmap.png}
}
\subfigure{
\includegraphics[width=.3\textwidth]{images/chapter6/vgg13/layers-g-16/0-tl31-Cam-On-Image.png}
}
\subfigure{
\includegraphics[width=.3\textwidth]{images/chapter6/vgg13/asbestos-original.png}
}
\subfigure{
\includegraphics[width=.3\textwidth]{images/chapter6/vgg13/layers-g-16/1-tl31-Cam-Heatmap.png}
}
\subfigure{
\includegraphics[width=.3\textwidth]{images/chapter6/vgg13/layers-g-16/1-tl31-Cam-On-Image.png}
}
\label{fig:asbestos_gradcam_g}
\end{figure}

The visualizations from Figure \ref{fig:asbestos_gradcam_g} need to be taken very critically. I suspect that modifying the architecture to that extent might have had an unexpected impact on the visualization library. Some visualizations of lower convolution layers don't succeed at all and yield unusable results.

As shown with the original model, Figure \ref{fig:vgg13_g_filter_activation} lists 4 layer visualizations performed from the last convolution layer from the modified VGG13 architecture. As already seen in chapter five, contrary to the pre-trained mode, no complex patterns can be found and that holds true for every filter throughout the model. In the appendix, all 40 filters of the architecture are shown for comparison.

\begin{figure}[!h]
\centering
\caption{Last Layer visualizations (4 of 4) of the VGG13\_16\_g architecture trained from scratch and with batch normalization.}
\subfigure{
\includegraphics[width=.23\textwidth]{images/chapter6/vgg13/layers-fs/l31-f0.jpg}
}
\subfigure{
\includegraphics[width=.23\textwidth]{images/chapter6/vgg13/layers-fs/l31-f1.jpg}
}
\subfigure{
\includegraphics[width=.23\textwidth]{images/chapter6/vgg13/layers-fs/l31-f2.jpg}
}
\subfigure{
\includegraphics[width=.23\textwidth]{images/chapter6/vgg13/layers-fs/l31-f3.jpg}
}
\label{fig:vgg13_g_filter_activation}
\end{figure}











\section{Evaluation of overall number of parameters in ResNet18}

Although VGG13 is the biggest network regarding the number of learnable parameters (over 128 million), ResNet18 in its original form also has quite many parameters (over 11 million). It has also been built to recognize 1'000 classes from ImageNet, which even distinguishes different breeds of dogs, and therefore is probably too big and too complex for the asbestos task. That is not to say, that the task is simple in its nature, but that asbestos has a common looking structure and the networks hard work is to find this structure in many different settings and under difficult conditions. Reducing the number of filters has another huge benefit of reducing the overall number of trainable parameters and thus speeding up the learning process, using less memory and being much easier to deploy on productive configurations.\\

For the above-mentioned reason, the ResNet18 implementation has been altered to contain only a certain amount of filters in every single layer. The amount has been set to different values ranging from 1 to 32 filters. With 32 filters of size 3x3, only 320 parameters are used per layer. In ResNet18, for example, there are 4 blocks of 2 layers of 320 parameters making in total 2560 parameters. That does not yet count in the first 7x7 convolution and the fully-connected layer with its softmax function but the number of parameters is reduced dramatically as shown in Table \ref{tbl:resnet18-different-filters}. The original version increases its layers from the initial 3 channels of an image to 64, 128, 256 and 512 filters up to the fully connected layer. The fully connected layer thus again makes a big bulk of all trainable parameters. Keeping the filters constant does reduce the number of trainable parameters within the network and automatically also reduces the parameters in the last fully connected layer.

Since observations on VGG13 have shown that pre-training is not helpful in this scenario, it has been omitted with ResNet18.


\begin{table}[!h] \centering
\ra{1.3}
\caption{Resnet18 with different number of filters on the FINAL dataset. The number of filters present in paranthesis is the number of filters used per layer.}
\resizebox{0.79\textwidth}{!}{%
\begin{tabular}{@{}rcccc@{}}
\toprule & Accuracy in \% (from scratch) & Trainable parameters &  \% of Original \\
\midrule
ResNet18 (official)      	& 79.62 	& 11'177'538 		& 100\%     \\
ResNet18 (32 filters)    	& 78.41 	& 156'578 		& 1.4008\%    \\
\textbf{ResNet18 (16 filters)}    	& \textbf{81.06} 	& \textbf{40'658} 			& \textbf{0.3637\%}    \\
ResNet18 (8 filters)      	& 79.40 	& 10'922 			& 0.0977\%     \\
ResNet18 (4 filters)      	& 79.62 	& 3'110 				& 0.0278\%     \\
\textbf{ResNet18 (2 filters)}      	& \textbf{80.07} 	& \textbf{968} 				& \textbf{0.0087\%}    \\
ResNet18 (1 filter)       	& 75.08 	& 338 				& 0.0030\%     \\
\bottomrule
\end{tabular}}
\label{tbl:resnet18-different-filters}
\end{table}

It is very surprising that reducing the learnable parameters by 99.64\% for the ResNet18 modification with 16 filters yields better results for the asbestos task trained from scratch. It does indeed confirm the hypothesis that for binary classification of a mineral the current models are way too big and wasteful. Also, the ResNet18 modification with only 2 filters per layer (reduction of parameters by 99.99\%) outperforms the original ResNet18 architecture.






\section{Evaluation of different dataset sizes and variations}

All the different dataset variations are compared against each other within the same architecture. Each result is obtained by running the model three times and averaging over all runs. DeepDIVA has a script that randomly distributes the images into one of the three sets. It is important to note, that once the sets have been created, the test set has never been altered. It stays the same for all dataset variations. Table \ref{tbl:resnet18_dataset} shows the results for ResNet18. \\

\begin{table}[!h] \centering
\ra{1.3}
\caption{Dataset variations with ResNet18. The first column shows how the datasets performed when trained from scratch whereas the second column shows how the datasets performed with pre-training.}
\resizebox{0.79\textwidth}{!}{%
\begin{tabular}{@{}rccc@{}}
\toprule & accuracy in \% (from scratch) & accuracy in \% (pre-trained) &   $\Delta$ \\
\midrule
FINAL                        	& 79.62  $\pm$ 0.95 	& 86.93 $\pm$ 0.78 		& +7.31  \\
FINAL\_C                   	& 80.73 $\pm$ 0.38 		& 84.05 $\pm$ 0.96 	& +3.32 \\
FINAL\_C\_B            	& 81.39 $\pm$ 1.38 		& 85.27 $\pm$ 0.80 		& +3.88 \\
FINAL\_CH                	& 81.62 $\pm$ 0.29 		& 84.94 $\pm$ 0.40 	& +3.32 \\
FINAL\_CH\_B           	& 80.73 $\pm$ 0.84 		& 85.05 $\pm$ 0.33 		& +4.32 \\
FINAL\_EXTENDED   	& 81.06 $\pm$ 0.38 		& 84.50 $\pm$ 0.86 	& +3.44 \\
\bottomrule
\end{tabular}}
\label{tbl:resnet18_dataset}
\end{table}

\quad

Changing the train set by reducing questionable and unclear images has lead to consistent improvements in the ResNet18 model trained from scratch but to consistently worse results when training from pre-trained weights. A possible explanation could be that training from scratch needs a higher quality of the datasets meaning no wrongly labeled images and no images with obvious errors in them. That would not explain why the model with transfer learning performs consistently worse with different training and validation sets. In Table \ref{tbl:densenet121_dataset} the summary shows how DenseNet121 reacts with different datasets. \\


\begin{table}[!h] \centering
\ra{1.3}
\caption{Dataset variations with DenseNet121. The first column shows how the datasets performed when trained from scratch whereas the second column shows how the datasets performed with pre-training.}
\resizebox{0.79\textwidth}{!}{%
\begin{tabular}{@{}rccc@{}}
\toprule & accuracy in \% (from scratch) & accuracy in \% (pre-trained) &   $\Delta$ \\
\midrule
FINAL                    		& 83.72 $\pm$ 1.20 		& 86.16 $\pm$ 1.17   	& +2.44     \\
FINAL\_C             		& 82.61 $\pm$ 0.97 		& 83.94 $\pm$ 0.59 	& +1.33 \\
FINAL\_C\_B        		& 80.39 $\pm$ 0.38 	& 85.38 $\pm$ 0.19 		& +4.99 \\
FINAL\_CH				& 79.61 $\pm$ 2.23 		& 86.60 $\pm$ 0.40 	& +6.99 \\
FINAL\_CH\_B          	& 83.17 $\pm$ 0.11		& 84.37 $\pm$ 0.81 		& +1.20 \\
FINAL\_EXTENDED	& 83.06 $\pm$ 1.03 		& 85.60 $\pm$ 2.12 		& +2.54 \\
\bottomrule
\end{tabular}}
\label{tbl:densenet121_dataset}
\end{table}

With Densenet121 there is no improvement to be seen by changing the dataset. If trained from scratch all variations lead to slightly worse results. If trained with pre-trained weights, the FINAL\_CH dataset yields marginally better accuracy but that is most likely due to chance. In Table \ref{tbl:inceptionv3_dataset} the summary shows how Inception v3 behaves with different datasets.

\begin{table}[!h] \centering
\ra{1.3}
\caption{Dataset variations with Inception v3. The first group shows how the datasets performed when trained from scratch whereas the second group shows how the datasets performed with pre-training. FINAL\_C\_B died for the non-pre-trained twice. Only one datapoint used.}
\resizebox{0.79\textwidth}{!}{%
\begin{tabular}{@{}rccc@{}}
\toprule & accuracy in \% (from scratch) & accuracy in \% (pre-trained) &   $\Delta$ \\
\midrule
FINAL                        & 82.83  $\pm$ 0.62 		& 85.93 $\pm$ 0.40  	& +3.10  \\
FINAL\_C                   & 80.4 $\pm$ 0.69 		& 85.82 $\pm$  0.29 	& +5.42 \\
FINAL\_C\_B              & 82.06 $\pm$ 1.77 		& 85.27 $\pm$  1.57 		& +3.21 \\
FINAL\_CH                 & 81.06 $\pm$ 0.54 		& 85.71 $\pm$ 0.27 		& +4.65 \\
FINAL\_CH\_B            & 81.06 $\pm$ 0.38 		& 85.27 $\pm$  1.22 		& +4.21 \\
FINAL\_EXTENDED   & 81.72 $\pm$ 0.38 		& 84.16 $\pm$ 0.44 		& +2.44 \\
\bottomrule
\end{tabular}}
\label{tbl:inceptionv3_dataset}
\end{table}

For the inception v3 architecture trained from scratch and trained with pre-trained weights, there is a very slight but constant disadvantage to using different datasets which I cannot make sense of. Removing images with errors or adding the validation images to the training set should increase overall accuracy, but this increase is never significantly observed.












\section{Evaluation of different cropping and augmentation methods}

In chapter 5, strong overfitting was observed in the training set. ResNet18 with transfer learning, for example, converged in only about 5 epochs to nearly 100\% accuracy on the training set and stayed there as seen in Figure \ref{fig:resnet18-graph}. The accuracy from the test set showed that this accuracy was due to overfitting. In order to mitigate this problem, several cropping and resizing methods will be used on ResNet18 with the goal of reducing overfitting on training and increasing generalization and thus achieving better results on the test set. Also, some randomness through random flipping and mirroring is introduced. As for the dataset, only the FINAL variation is used for this part.

\subsection{Randomness}

Here, random cropping is performed first in order to crop the image to the smallest image height in the dataset which is 768 pixels this already allows vertical and horizontal randomness leading to every epoch training on different augmented images. Then the image is resized to the expected input size of the network being 224x224 pixels for ResNet18. With resizing a  smaller image then the original image, less information is lost due to the resizing process, which is a good thing. In the end, two transformations happen with a probability of 0.5: flipping horizontally and flipping vertically. All transformations are performed on the training set only. One run was performed as usual with 50 epochs and a  learning rate decay every 20 epochs. Since data augmentation yields new images to train from every epoch, there is much more variation and less overfitting. Therefore another 3 runs have been performed with 100 epochs and a learning rate decay every  30 epochs. The summary of these is shown in Table \ref{tbl:ResNet18_cropping}. \\


\begin{table}[!h] \centering
\ra{1.3}
\caption{Cropping, resizing and  random flipping performed on the ResNet18 architecture with and without transfer learning. The first row is a run with the original configuration without any randomness or cropping and serves as comparison.}
\resizebox{0.79\textwidth}{!}{%
\begin{tabular}{@{}rrrr@{}}
\toprule & Accuracy in \% (from scratch) & Accuracy in \% (pre-trained) &   $\Delta$ in \% \\
\midrule
ResNet18 original & 79.62  $\pm$ 0.95 &  85.83 $\pm$ 0.41 & +6.21     \\
\midrule
ResNet18 (50 epochs) & 81.73  $\pm$ 1.09 &  84.83 $\pm$ 0.78 & +3.10     \\
ResNet18 (100 epochs) &  81.17 $\pm$ 0.87 & 85.93 $\pm$ 0.32 & +4.76 \\
\bottomrule
\end{tabular}}
\label{tbl:ResNet18_cropping}
\end{table}

Unfortunately, the improvement in accuracy of only 2.11\% for the ResNet18 trained from scratch with simple data augmentation is not as high as expected. From the model with the pre-trained weights, it even gets worse with a decrease of exactly 1\%. But as seen in Figure \ref{fig:resnet18-da-graph} no overfitting takes place for the model trained from scratch and less overfitting is found in the pre-trained model. This is in strong contrast to Figure \ref{fig:resnet18-graph} from Chapter 5.

\begin{figure}[!h]
\centering
\caption{Training and validation graphs for ResNet18 with simple data augmentation. Once with pre-trained weights from ImageNet and once trained from scratch.}
\subfigure{
\includegraphics[width=.46\textwidth]{images/chapter6/resnet18-da/TA-resnet18-da.png}
}
\subfigure{
\includegraphics[width=.46\textwidth]{images/chapter6/resnet18-da/VA-resnet18-da.png}
}
\label{fig:resnet18-da-graph}
\end{figure}

Convergence happens at roughly the same amount of epochs. The model trained from scratch does never catch up with the pre-trained model which has already been attributed to the more complex feature mappings of the pre-trained model. A very interesting point can be seen in Table \ref{tbl:ResNet18_cropping_different_acc} where on the first two rows the ResNet runs are shown as performed in Chapter 5. Severe overfitting happens for the pre-trained model after only a few epochs and after the model trained from scratch towards the end. But both are near 100\% accuracy for the training. The validation accuracy is much more realistic since it does only test the network on this data. Validation and test set accuracies should be near each other. For the ResNet18 architecture with cropping and 50 epochs no overfitting can be observed. The training accuracy remains around 82\% for training, validation and test sets. This simple data augmentation is enough to create new images for roughly 50 epochs, but not for 100 epochs. Although there is some randomness introduced, it is not sophisticated enough.

\begin{table}[!h] \centering
\ra{1.3}
\caption{Different set accuracies. If the training accuracy is much higher than the validation or test accuracy, then there is a problem with overfitting to the training data.}
\resizebox{0.85\textwidth}{!}{%
\begin{tabular}{@{}rccc@{}}
\toprule & Accuracy train set in \% & Accuracy validation set in \% &  Accuracy test set in \% \\
\midrule
ResNet18 original (from scratch)			& 99.28   		&  77.83 	& 79.62     \\
ResNet18 original (pre-trained)				& 100.00   	&  88.61 	& 85.83     \\
\midrule
ResNet18 (50 epochs / from scratch)	& 83.33   		&  82.25 	& 81.73   \\
ResNet18 (50 epochs / pre-trained)		& 95.85   		&  88.57 	& 84.83     \\
\midrule
ResNet18 (100 epochs / from scratch)	& 84.70   		&  83.22 	& 81.17     \\
ResNet18 (100 epochs / pre-trained)	 	&  98.07 		& 87.57 	& 85.93 \\
\bottomrule
\end{tabular}}
\label{tbl:ResNet18_cropping_different_acc}
\end{table}

\subsection{FiveCrop}

The Torch library comes with a FiveCrop implementation which could be used with minor changes in the code. The FiveCrop implementation crops 5 images of a given size from the whole image. All corners and the center are cropped. In training and evaluation, these crops are stacked on top of each other and fed to the network. The output is averaged across all crops before the weights get updated. Table \ref{tbl:resnet18-fivecrop} summarizes the results.\\


\begin{table}[!h] \centering
\ra{1.3}
\caption{Resnet18 FiveCrop Implementation with and without pre-training. FINAL (regular) means ResNet18 with the resizing of the image instead of cropping and averaging}
\resizebox{0.79\textwidth}{!}{%
\begin{tabular}{@{}rccc@{}}
\toprule & Accuracy in \% (from scratch) & Accuracy in \% (pre-trained) &   $\Delta$ in \%  \\
\midrule
FINAL (regular)          	& 79.62  $\pm$ 0.95 	& 86.93 $\pm$ 0.78   	& +7.31     \\
FINAL (fiveCrop)          	& 82.28 $\pm$ 1.81 		& 87.93 $\pm$ 1.49 		& +5.65   \\
\midrule
Accuracy gain				& +2.66						& +1.00 						& - \\
\bottomrule
\end{tabular}}
\label{tbl:resnet18-fivecrop}
\end{table}

With FiveCrop slightly better results are achieved than without data augmentation. The increase for the model trained from scratch is 2.66\% whereas the increase for the pre-trained model is 1\%. No overfitting is taking place. The training, validation and test accuracies are all around 83\% for the model trained from scratch. For the pre-trained model, the training accuracy is around 91\% which is slightly overfitting. Validation and test accuracies are both around 88\%.

\subsection{NineCrop}

The NineCrop is an own implementation that puts more weight on the center of the image. The nine crops taken from the image are all symmetrically positioned around the center with only minimal or none overlap. Taking nine crops instead of five is not without disadvantages. If many crops end up having no asbestos in them it will rather harm than improve the learning process. Table \ref{tbl:resnet18-randomnine} summarizes the results.

\begin{table}[!h] \centering
\ra{1.3}
\caption{Resnet18 NineCrop implementation with and without pre-training. FINAL (regular) means ResNet18 with the resizing of the image instead of cropping and averaging}
\resizebox{0.79\textwidth}{!}{%
\begin{tabular}{@{}rccc@{}}
\toprule & Accuracy in \% (from scratch) & Accuracy in \% (pre-trained) &   $\Delta$ in \% \\
\midrule
FINAL (regular)              		& 79.62  $\pm$ 0.95 	& 86.93 $\pm$ 0.78		& +7.31    \\
FINAL (randomNine)          & 83.39 $\pm$ 0.95 		& 87.92 $\pm$ 1.13 		& +4.53    \\
\midrule
Accuracy gain					& +3.77							& +1.00 						& - \\
\bottomrule
\end{tabular}}
\label{tbl:resnet18-randomnine}
\end{table}

As seen in Figure \ref{fig:resnet18-ninecrop-graph} almost no overfitting takes place. The  training accuracy of the pre-trained model reaches values just sligthly above  90\%  but never wiggles around 100\% accuracy as in the models without data augentation. The model trained from scratch has no overfitting whatsoever.

\begin{figure}[!h]
\centering
\caption{Training and validation graphs for ResNet18 with NineCrop implementation. Once with pre-trained weights from ImageNet and once trained from scratch.}
\subfigure{
\includegraphics[width=.46\textwidth]{images/chapter6/ninecrop/TrainingAccuracyNineCrop.png}
}
\subfigure{
\includegraphics[width=.46\textwidth]{images/chapter6/ninecrop/ValidationAccuracyNineCrop.png}
}
\label{fig:resnet18-ninecrop-graph}
\end{figure}

\section{Evaluation of different image size Inputs to the network}

The input size of the network ranges relative to the architectures from 224x224 pixels for ResNet to 299x299 pixels for inception v3. Resizing the image inevitably leads to a loss of information. Therefore one path of investigation was to check if it is possible to change the architecture in a certain way, that allows for bigger images to be fed into the network. Images are highly compressed in jpg or png format but when they are loaded into tensors or arrays they increase their size drastically which leads to the problem of not having enough memory. Therefore, the ResNet18 architecture with 16 filters throughout the network has been used, to allow for bigger images. No transfer learning could be applied for that reason. In order to accommodate bigger input images into the network, convolutional layers have been added on top of the network that halves the input volume and prepares the image for the previous first layer of 224x224 pixels. For the ResNet18\_448 modification the added code is shown in the following listing:

\begin{minipage}{\linewidth}
\begin{lstlisting}[language=Python, caption=Python example, basicstyle=\tiny]
        self.expected_input_size = (448, 448)

        # ************************************************************************************************
        self.conv00 = nn.Conv2d(3, constant_number_of_filters, kernel_size=7, stride=2, padding=3,
                                bias=False)
        self.bn00 = nn.BatchNorm2d(constant_number_of_filters)
        # ************************************************************************************************
\end{lstlisting}
\end{minipage}

The forward pass consists of doing first the new convolution with batch normalization, then passing the output through a ReLU and then feeding it to the previously first convolution layer that expects the original size. Table \ref{tbl:resnet18-different-input} summarizes the results.

\begin{table}[!h] \centering
\ra{1.3}
\caption{Different image input sizes are fed into ResNet18 architectures with only 16 filters per layer. The first layers of the network have been adapted to allow bigger input images}
\resizebox{0.79\textwidth}{!}{%
\begin{tabular}{@{}rcc@{}}
\toprule & Accuracy in \% (from scratch) & Accuracy in \% (pre-trained) \\
\midrule
FINAL (regular 224)     	& 79.62  $\pm$ 0.95 & 86.93\% $\pm$ 0.7757     \\
FINAL (input 448)       	& 77.41 $\pm$ 1.69 & -     \\
FINAL (input 896)        	& 79.62 $\pm$ 0.68 & -     \\
FINAL (input 1024)       	& 80.62 $\pm$ 0.83 & -     \\
\bottomrule
\end{tabular}}
\label{tbl:resnet18-different-input}
\end{table}

Increasing the input size from 224x224 pixels to 448x448 pixels decreased the overall accuracy by roughly 2\%. Increasing it to 896x896 achieved the exact same accuracy with a lower standard deviation and increasing it to the full size of the image increased the accuracy by almost 1\%. Overall I expected better results but all the networks have been carefully crafted by their creators and simply adding convolutions on top of it does alter the network in quite unpredictable ways. For the ResNet18\_1024 modification I had to add 8 convolution layers and 8 batch normalization in order to get the input volume down to 224x224x16 (with 16 filters) which looks like this:

\begin{minipage}{\linewidth}
\begin{lstlisting}[language=Python, caption=Python example, basicstyle=\tiny]
self.expected_input_size = (1024, 1024)

        # ************************************************************************************************
        self.conv01 = nn.Conv2d(3, constant_number_of_filters, kernel_size=7, stride=2,
                                padding=3, bias=False)
        self.bn01 = nn.BatchNorm2d(constant_number_of_filters)

        self.conv02 = nn.Conv2d(constant_number_of_filters, constant_number_of_filters, kernel_size=7, stride=2,
                                padding=3, bias=False)
        self.bn02 = nn.BatchNorm2d(constant_number_of_filters)

        # 256
        self.conv03 = nn.Conv2d(constant_number_of_filters, constant_number_of_filters, kernel_size=7, stride=1,
                                padding=0, bias=False)
        self.bn03 = nn.BatchNorm2d(constant_number_of_filters)

        # 250

        self.conv04 = nn.Conv2d(constant_number_of_filters, constant_number_of_filters, kernel_size=7, stride=1,
                                padding=0, bias=False)
        self.bn04 = nn.BatchNorm2d(constant_number_of_filters)

        # 244

        self.conv05 = nn.Conv2d(constant_number_of_filters, constant_number_of_filters, kernel_size=7, stride=1,
                                padding=0, bias=False)
        self.bn05 = nn.BatchNorm2d(constant_number_of_filters)

        # 238

        self.conv06 = nn.Conv2d(constant_number_of_filters, constant_number_of_filters, kernel_size=7, stride=1,
                                padding=0, bias=False)
        self.bn06 = nn.BatchNorm2d(constant_number_of_filters)

        # 232

        self.conv07 = nn.Conv2d(constant_number_of_filters, constant_number_of_filters, kernel_size=7, stride=1,
                                padding=0, bias=False)
        self.bn07 = nn.BatchNorm2d(constant_number_of_filters)

        # 226

        self.conv08 = nn.Conv2d(constant_number_of_filters, constant_number_of_filters, kernel_size=3, stride=1,
                                padding=0, bias=False)
        self.bn08 = nn.BatchNorm2d(constant_number_of_filters)

        # 224

        # ************************************************************************************************
\end{lstlisting}
\end{minipage}

This makes it so much harder to train this network since there are 9 convolution layers before the actual ResNet18 architecture starts. There are no skip connections in the first 9 layers and therefore the vanishing/exploding gradient problem starts to play a role again.


\section{Conclusion}

Several modifications have been performed on VGG13 and ResNet18 architectures in order to squeeze out more accuracy on the test set. VGG13's last layer got reduced to 16 units creating a much smaller architecture of only 7.61\% of its original size, increasing accuracy to 82.06\%. Then going through different filter schema it was observed, that further reduction of the parameters is still possible with only a minor decrease in accuracy. Overall it may be said that the original architectures were built for discriminating between 1'000 classes some of which were very similar to each other. For the asbestos recognition task that is clearly unnecessary.

Reducing the overall number of parameters makes transfer learning not applicable anymore since most of the useful filters are lost due to the reduction. The same behavior was observed with ResNet18 where a reduction by 99.99\% of the learnable parameters still leads to the same accuracy. Further decrease leads to worse accuracies.

Variations in the dataset didn't yield the expected boost in performance. It was hoped that removing clear errors from the dataset would make the training process easier but that was not the case. What surprised, even more, was that the extension of the dataset by adding the validation images to the training images didn't improve the overall performance. This could be due to the images not being correctly classified and the dataset quality that does not allow for better performance since the 90\% mark was never broken on a regular basis regardless of all the changes.

Data augmentation was able to reduce overfitting in a regular 50 epoch run and lead to marginally better performance. Training, validation and test sets all reached roughly the same accuracy which was the goal of this task. The boost in performance through better generalization was not achieved as hoped for.

Modifying the architecture in order to allow for bigger image input did lead to a marginal improvement only for the last modification where the full image was fed into the network. Unfortunately, the filters had to be reduced to only 16 for such big images to process making transfer learning unavailable.


\chapter{Conclusion}

\section{Summary and Conclusion}

Contrary to some believes that transfer learning is not necessarily a good option if using across domains, this work shows that even for microscopic images with one object being looked for, pre-training on ImageNet is a good option. Overall, transfer learning led to improved accuracies by roughly 8\% and often yielded faster convergence. Especially in the asbestos task, where only about 1'000 (around 500 per class) images were used for training, transfer learning helped a lot. It was surprising though, that even when the models were run 3 times longer from scratch, they did usually not catch up with the pre-trained models. Visualizations of the layer activations have shown that when the model is trained from scratch no complex (mid-level and high-level) patterns are learned. I argue that is due to the less than ideal quality and size of the dataset. Pre-training provided some general purpose filters that were applicable to the asbestos task, thus leading to better accuracy.

Visualizing the layers and creating heatmaps was much more difficult than expected. Finding libraries was difficult and implementing them in PyTorch was a struggle. Every architecture has its own model representation which would need transforming it into another format, known by the library. But the visualization on VGG13 was not as insightful as initially hoped for. Maybe it was due to the network simply not learning any better features. Heatmaps were also difficult to interpret since different layers showed partially contradicting areas of interest for a certain class.

Cropping was especially difficult since the asbestos fibers were not spaced evenly over the image. So cropping would always lead to some crops with and some crops without asbestos. Feeding all the crops to the network and averaging its output before updating the weights, can only lead to better performance when more crops include asbestos than not. Still, cropping with somehow retaining the correct label is still the way to go since variation and randomness are introduced and the small size of the dataset is efficiently mitigated.

Since the quality and size of the dataset is clearly a problem, I tried to change it in certain ways in order to achieve higher quality. One variation of the dataset removed all very unclear or faulty images from training. Another time I reduced it even more to contain only clear and good examples. My intention was to give the network high-quality example images to learn from correctly and then apply the knowledge on all the images, unclear images included. Another time I tried to extend the training set by adding all validation images to the training. Although sometimes slightly better performance was achieved, it wasn't really significant.

Modifications to the network were like a drastic reduction of the learnable parameters by reducing filters and units in the fully connected layers lead to the same accuracies or even increased accuracies by a small amount. More interestingly was that a reduction of the complexity of the network by roughly 99\% had no detrimental effect.

All in all, I think that the quality of the current dataset is the limiting factor since all the architectures reach values between 80\% and 90\% and never cross the 90\% threshold regardless of different cropping and data augmentation methods, different dataset considerations and different modifications to the architectures. The accuracies stay more or less the same although I was able to achieve parameter reductions of 99.99\% which makes the asbestos recognition task easily deployable and efficient.

\section{Future Work}

As mentioned in several papers, the pre-processing of the image itself might play a vital role in the ease of asbestos detection. For example, thresholding and binarization reduce the noise in the image and transform a grayscale image into a black and white image. This allows to clearly identify the asbestos-like structures and could potentially lead to architectures with fewer parameters and better performance. Although Deep Learning architectures should extract the needed features by themselves, as seen with the visualizations that are not necessarily the case with only a few images. In that specific case, pre-processing of the images could help.\\

The dataset quality is of utmost importance. Future work could be channeled into producing a much bigger high-quality dataset with good images and correct labels.\\

Continued work on the visualizations could be very rewarding. Especially making the visual toolbox work for other architectures and to visualize models that were trained only on grayscale colors instead of RGB. Training on grayscale colors (having only one channel in the input volume) resulted in the same accuracies as training with RGB. I would expect to have more interesting visualizations and less noise when focusing on grayscale values.\\

Reducing the number of parameters by over 99\% did not harm the performance and even improved in some cases. Transfer learning cannot be used in such a scenario since most of the feature mappings would be reduced as well. Being able to pinpoint the filters that lead to the highest activations and transfer only them to the remaining filters in the reduced architectures, could lead to better performance and less complexity.\\

\chapter{Appendix}


%%% CNN layer vis from pre-trained model
%%%%%%%%%%%%%%%%%%%%%%%%%%

\begin{figure}[H]
\centering
\caption{Each line represents one convolution layer in the VGG13\_bn architecture is ascending order from a pre-trained model. This Figure shows the first 4 convolution layer visualization, from which 4 interesting images were selected.}

\subfigure{
\includegraphics[width=.23\textwidth]{images/chapter5/vgg13-bn-pre/l0-f0.jpg}
}
\subfigure{
\includegraphics[width=.23\textwidth]{images/chapter5/vgg13-bn-pre/l0-f1.jpg}
}
\subfigure{
\includegraphics[width=.23\textwidth]{images/chapter5/vgg13-bn-pre/l0-f2.jpg}
}
\subfigure{
\includegraphics[width=.23\textwidth]{images/chapter5/vgg13-bn-pre/l0-f3.jpg}
}

\subfigure{
\includegraphics[width=.23\textwidth]{images/chapter5/vgg13-bn-pre/l3-f0.jpg}
}
\subfigure{
\includegraphics[width=.23\textwidth]{images/chapter5/vgg13-bn-pre/l3-f1.jpg}
}
\subfigure{
\includegraphics[width=.23\textwidth]{images/chapter5/vgg13-bn-pre/l3-f2.jpg}
}
\subfigure{
\includegraphics[width=.23\textwidth]{images/chapter5/vgg13-bn-pre/l3-f3.jpg}
}

\subfigure{
\includegraphics[width=.23\textwidth]{images/chapter5/vgg13-bn-pre/l7-f0.jpg}
}
\subfigure{
\includegraphics[width=.23\textwidth]{images/chapter5/vgg13-bn-pre/l7-f1.jpg}
}
\subfigure{
\includegraphics[width=.23\textwidth]{images/chapter5/vgg13-bn-pre/l7-f2.jpg}
}
\subfigure{
\includegraphics[width=.23\textwidth]{images/chapter5/vgg13-bn-pre/l7-f3.jpg}
}

\subfigure{
\includegraphics[width=.23\textwidth]{images/chapter5/vgg13-bn-pre/l10-f0.jpg}
}
\subfigure{
\includegraphics[width=.23\textwidth]{images/chapter5/vgg13-bn-pre/l10-f1.jpg}
}
\subfigure{
\includegraphics[width=.23\textwidth]{images/chapter5/vgg13-bn-pre/l10-f2.jpg}
}
\subfigure{
\includegraphics[width=.23\textwidth]{images/chapter5/vgg13-bn-pre/l10-f3.jpg}
}
\label{fig:vgg13_app_pretrained_filters_a}
\end{figure}



\begin{figure}[H]
\centering
\caption{Each line represents one convolution layer in the VGG13\_bn architecture is ascending order from a pre-trained model. This Figure shows the next 4 convolution layer visualization, from which 4 interesting images were selected.}
\subfigure{
\includegraphics[width=.23\textwidth]{images/chapter5/vgg13-bn-pre/l14-f0.jpg}
}
\subfigure{
\includegraphics[width=.23\textwidth]{images/chapter5/vgg13-bn-pre/l14-f1.jpg}
}
\subfigure{
\includegraphics[width=.23\textwidth]{images/chapter5/vgg13-bn-pre/l14-f2.jpg}
}
\subfigure{
\includegraphics[width=.23\textwidth]{images/chapter5/vgg13-bn-pre/l14-f3.jpg}
}


\subfigure{
\includegraphics[width=.23\textwidth]{images/chapter5/vgg13-bn-pre/l17-f0.jpg}
}
\subfigure{
\includegraphics[width=.23\textwidth]{images/chapter5/vgg13-bn-pre/l17-f1.jpg}
}
\subfigure{
\includegraphics[width=.23\textwidth]{images/chapter5/vgg13-bn-pre/l17-f2.jpg}
}
\subfigure{
\includegraphics[width=.23\textwidth]{images/chapter5/vgg13-bn-pre/l17-f3.jpg}
}

\subfigure{
\includegraphics[width=.23\textwidth]{images/chapter5/vgg13-bn-pre/l21-f0.jpg}
}
\subfigure{
\includegraphics[width=.23\textwidth]{images/chapter5/vgg13-bn-pre/l21-f1.jpg}
}
\subfigure{
\includegraphics[width=.23\textwidth]{images/chapter5/vgg13-bn-pre/l21-f2.jpg}
}
\subfigure{
\includegraphics[width=.23\textwidth]{images/chapter5/vgg13-bn-pre/l21-f3.jpg}
}

\subfigure{
\includegraphics[width=.23\textwidth]{images/chapter5/vgg13-bn-pre/l24-f0.jpg}
}
\subfigure{
\includegraphics[width=.23\textwidth]{images/chapter5/vgg13-bn-pre/l24-f1.jpg}
}
\subfigure{
\includegraphics[width=.23\textwidth]{images/chapter5/vgg13-bn-pre/l24-f2.jpg}
}
\subfigure{
\includegraphics[width=.23\textwidth]{images/chapter5/vgg13-bn-pre/l24-f3.jpg}
}
\label{fig:vgg13_app_pretrained_filters_b}
\end{figure}


\begin{figure}[H]
\centering
\caption{Each line represents one convolution layer in the VGG13\_bn architecture is ascending order from a pre-trained model. This Figure shows the last 2 convolution layer visualization, from which 4 interesting images were selected.}

\subfigure{
\includegraphics[width=.23\textwidth]{images/chapter5/vgg13-bn-pre/l28-f0.jpg}
}
\subfigure{
\includegraphics[width=.23\textwidth]{images/chapter5/vgg13-bn-pre/l28-f1.jpg}
}
\subfigure{
\includegraphics[width=.23\textwidth]{images/chapter5/vgg13-bn-pre/l28-f2.jpg}
}
\subfigure{
\includegraphics[width=.23\textwidth]{images/chapter5/vgg13-bn-pre/l28-f3.jpg}
}

\subfigure{
\includegraphics[width=.23\textwidth]{images/chapter5/vgg13-bn-pre/l31-f0.jpg}
}
\subfigure{
\includegraphics[width=.23\textwidth]{images/chapter5/vgg13-bn-pre/l31-f1.jpg}
}
\subfigure{
\includegraphics[width=.23\textwidth]{images/chapter5/vgg13-bn-pre/l31-f2.jpg}
}
\subfigure{
\includegraphics[width=.23\textwidth]{images/chapter5/vgg13-bn-pre/l31-f3.jpg}
}

\label{fig:vgg13_app_pretrained_filters_c}
\end{figure}












%%% CNN layer vis from model trained from scratch
%%%%%%%%%%%%%%%%%%%%%%%%%%%%%%



\begin{figure}[H]
\centering
\caption{Each line represents one convolution layer in the VGG13\_bn architecture is ascending order of a model trained from scratch. This Figure shows the first 5 convolution layer visualization, from which 4 interesting images were selected.}
\subfigure{
\includegraphics[width=.23\textwidth]{images/chapter5/vgg13-bn/l0-f0.jpg}
}
\subfigure{
\includegraphics[width=.23\textwidth]{images/chapter5/vgg13-bn/l0-f1.jpg}
}
\subfigure{
\includegraphics[width=.23\textwidth]{images/chapter5/vgg13-bn/l0-f2.jpg}
}
\subfigure{
\includegraphics[width=.23\textwidth]{images/chapter5/vgg13-bn/l0-f3.jpg}
}

\subfigure{
\includegraphics[width=.23\textwidth]{images/chapter5/vgg13-bn/l3-f0.jpg}
}
\subfigure{
\includegraphics[width=.23\textwidth]{images/chapter5/vgg13-bn/l3-f1.jpg}
}
\subfigure{
\includegraphics[width=.23\textwidth]{images/chapter5/vgg13-bn/l3-f2.jpg}
}
\subfigure{
\includegraphics[width=.23\textwidth]{images/chapter5/vgg13-bn/l3-f3.jpg}
}

\subfigure{
\includegraphics[width=.23\textwidth]{images/chapter5/vgg13-bn/l7-f0.jpg}
}
\subfigure{
\includegraphics[width=.23\textwidth]{images/chapter5/vgg13-bn/l7-f1.jpg}
}
\subfigure{
\includegraphics[width=.23\textwidth]{images/chapter5/vgg13-bn/l7-f2.jpg}
}
\subfigure{
\includegraphics[width=.23\textwidth]{images/chapter5/vgg13-bn/l7-f3.jpg}
}

\subfigure{
\includegraphics[width=.23\textwidth]{images/chapter5/vgg13-bn/l10-f0.jpg}
}
\subfigure{
\includegraphics[width=.23\textwidth]{images/chapter5/vgg13-bn/l10-f1.jpg}
}
\subfigure{
\includegraphics[width=.23\textwidth]{images/chapter5/vgg13-bn/l10-f2.jpg}
}
\subfigure{
\includegraphics[width=.23\textwidth]{images/chapter5/vgg13-bn/l10-f3.jpg}
}

\subfigure{
\includegraphics[width=.23\textwidth]{images/chapter5/vgg13-bn/l14-f0.jpg}
}
\subfigure{
\includegraphics[width=.23\textwidth]{images/chapter5/vgg13-bn/l14-f1.jpg}
}
\subfigure{
\includegraphics[width=.23\textwidth]{images/chapter5/vgg13-bn/l14-f2.jpg}
}
\subfigure{
\includegraphics[width=.23\textwidth]{images/chapter5/vgg13-bn/l14-f3.jpg}
}

\label{fig:vgg13_app_fromscratch_filters_a}
\end{figure}


\begin{figure}[H]
\centering
\caption{Each line represents one convolution layer in the VGG13\_bn architecture is ascending order of a model trained from scratch. This Figure shows the last 5 convolution layer visualization, from which 4 interesting images were selected.}

\subfigure{
\includegraphics[width=.23\textwidth]{images/chapter5/vgg13-bn/l17-f0.jpg}
}
\subfigure{
\includegraphics[width=.23\textwidth]{images/chapter5/vgg13-bn/l17-f1.jpg}
}
\subfigure{
\includegraphics[width=.23\textwidth]{images/chapter5/vgg13-bn/l17-f2.jpg}
}
\subfigure{
\includegraphics[width=.23\textwidth]{images/chapter5/vgg13-bn/l17-f3.jpg}
}

\subfigure{
\includegraphics[width=.23\textwidth]{images/chapter5/vgg13-bn/l21-f0.jpg}
}
\subfigure{
\includegraphics[width=.23\textwidth]{images/chapter5/vgg13-bn/l21-f1.jpg}
}
\subfigure{
\includegraphics[width=.23\textwidth]{images/chapter5/vgg13-bn/l21-f2.jpg}
}
\subfigure{
\includegraphics[width=.23\textwidth]{images/chapter5/vgg13-bn/l21-f3.jpg}
}

\subfigure{
\includegraphics[width=.23\textwidth]{images/chapter5/vgg13-bn/l24-f0.jpg}
}
\subfigure{
\includegraphics[width=.23\textwidth]{images/chapter5/vgg13-bn/l24-f1.jpg}
}
\subfigure{
\includegraphics[width=.23\textwidth]{images/chapter5/vgg13-bn/l24-f2.jpg}
}
\subfigure{
\includegraphics[width=.23\textwidth]{images/chapter5/vgg13-bn/l24-f3.jpg}
}

\subfigure{
\includegraphics[width=.23\textwidth]{images/chapter5/vgg13-bn/l28-f0.jpg}
}
\subfigure{
\includegraphics[width=.23\textwidth]{images/chapter5/vgg13-bn/l28-f1.jpg}
}
\subfigure{
\includegraphics[width=.23\textwidth]{images/chapter5/vgg13-bn/l28-f2.jpg}
}
\subfigure{
\includegraphics[width=.23\textwidth]{images/chapter5/vgg13-bn/l28-f3.jpg}
}

\subfigure{
\includegraphics[width=.23\textwidth]{images/chapter5/vgg13-bn/l31-f0.jpg}
}
\subfigure{
\includegraphics[width=.23\textwidth]{images/chapter5/vgg13-bn/l31-f1.jpg}
}
\subfigure{
\includegraphics[width=.23\textwidth]{images/chapter5/vgg13-bn/l31-f2.jpg}
}
\subfigure{
\includegraphics[width=.23\textwidth]{images/chapter5/vgg13-bn/l31-f3.jpg}
}

\label{fig:vgg13_app_fromscratch_filters_b}
\end{figure}









\begin{figure}[H]
\centering
\caption{Layer visualization of all the possible filters in the first 5 layers in the modified VGG13 architecture. This architecture uses filter sheme "G" with only 16 units in the last fully connected layers, reducing the parameter amount to 0.0039\%}
\subfigure{
\includegraphics[width=.23\textwidth]{images/chapter6/vgg13/layers-fs/l0-f0.jpg}
}
\subfigure{
\includegraphics[width=.23\textwidth]{images/chapter6/vgg13/layers-fs/l0-f1.jpg}
}
\subfigure{
\includegraphics[width=.23\textwidth]{images/chapter6/vgg13/layers-fs/l0-f2.jpg}
}
\subfigure{
\includegraphics[width=.23\textwidth]{images/chapter6/vgg13/layers-fs/l0-f3.jpg}
}

\subfigure{
\includegraphics[width=.23\textwidth]{images/chapter6/vgg13/layers-fs/l3-f0.jpg}
}
\subfigure{
\includegraphics[width=.23\textwidth]{images/chapter6/vgg13/layers-fs/l3-f1.jpg}
}
\subfigure{
\includegraphics[width=.23\textwidth]{images/chapter6/vgg13/layers-fs/l3-f2.jpg}
}
\subfigure{
\includegraphics[width=.23\textwidth]{images/chapter6/vgg13/layers-fs/l3-f3.jpg}
}

\subfigure{
\includegraphics[width=.23\textwidth]{images/chapter6/vgg13/layers-fs/l7-f0.jpg}
}
\subfigure{
\includegraphics[width=.23\textwidth]{images/chapter6/vgg13/layers-fs/l7-f1.jpg}
}
\subfigure{
\includegraphics[width=.23\textwidth]{images/chapter6/vgg13/layers-fs/l7-f2.jpg}
}
\subfigure{
\includegraphics[width=.23\textwidth]{images/chapter6/vgg13/layers-fs/l7-f3.jpg}
}

\subfigure{
\includegraphics[width=.23\textwidth]{images/chapter6/vgg13/layers-fs/l10-f0.jpg}
}
\subfigure{
\includegraphics[width=.23\textwidth]{images/chapter6/vgg13/layers-fs/l10-f1.jpg}
}
\subfigure{
\includegraphics[width=.23\textwidth]{images/chapter6/vgg13/layers-fs/l10-f2.jpg}
}
\subfigure{
\includegraphics[width=.23\textwidth]{images/chapter6/vgg13/layers-fs/l10-f3.jpg}
}


\subfigure{
\includegraphics[width=.23\textwidth]{images/chapter6/vgg13/layers-fs/l14-f0.jpg}
}
\subfigure{
\includegraphics[width=.23\textwidth]{images/chapter6/vgg13/layers-fs/l14-f1.jpg}
}
\subfigure{
\includegraphics[width=.23\textwidth]{images/chapter6/vgg13/layers-fs/l14-f2.jpg}
}
\subfigure{
\includegraphics[width=.23\textwidth]{images/chapter6/vgg13/layers-fs/l14-f3.jpg}
}
\label{fig:vgg13_g_filter_activation_appendix_a}
\end{figure}



\begin{figure}[H]
\centering
\caption{Layer visualization of all the possible filters in the last 5 layers in the modified VGG13 architecture. This architecture uses filter sheme "G" with only 16 units in the last fully connected layers, reducing the parameter amount to 0.0039\%}
\subfigure{
\includegraphics[width=.23\textwidth]{images/chapter6/vgg13/layers-fs/l17-f0.jpg}
}
\subfigure{
\includegraphics[width=.23\textwidth]{images/chapter6/vgg13/layers-fs/l17-f1.jpg}
}
\subfigure{
\includegraphics[width=.23\textwidth]{images/chapter6/vgg13/layers-fs/l17-f2.jpg}
}
\subfigure{
\includegraphics[width=.23\textwidth]{images/chapter6/vgg13/layers-fs/l17-f3.jpg}
}

\subfigure{
\includegraphics[width=.23\textwidth]{images/chapter6/vgg13/layers-fs/l21-f0.jpg}
}
\subfigure{
\includegraphics[width=.23\textwidth]{images/chapter6/vgg13/layers-fs/l21-f1.jpg}
}
\subfigure{
\includegraphics[width=.23\textwidth]{images/chapter6/vgg13/layers-fs/l21-f2.jpg}
}
\subfigure{
\includegraphics[width=.23\textwidth]{images/chapter6/vgg13/layers-fs/l21-f3.jpg}
}

\subfigure{
\includegraphics[width=.23\textwidth]{images/chapter6/vgg13/layers-fs/l24-f0.jpg}
}
\subfigure{
\includegraphics[width=.23\textwidth]{images/chapter6/vgg13/layers-fs/l24-f1.jpg}
}
\subfigure{
\includegraphics[width=.23\textwidth]{images/chapter6/vgg13/layers-fs/l24-f2.jpg}
}
\subfigure{
\includegraphics[width=.23\textwidth]{images/chapter6/vgg13/layers-fs/l24-f3.jpg}
}

\subfigure{
\includegraphics[width=.23\textwidth]{images/chapter6/vgg13/layers-fs/l28-f0.jpg}
}
\subfigure{
\includegraphics[width=.23\textwidth]{images/chapter6/vgg13/layers-fs/l28-f1.jpg}
}
\subfigure{
\includegraphics[width=.23\textwidth]{images/chapter6/vgg13/layers-fs/l28-f2.jpg}
}
\subfigure{
\includegraphics[width=.23\textwidth]{images/chapter6/vgg13/layers-fs/l28-f3.jpg}
}


\subfigure{
\includegraphics[width=.23\textwidth]{images/chapter6/vgg13/layers-fs/l31-f0.jpg}
}
\subfigure{
\includegraphics[width=.23\textwidth]{images/chapter6/vgg13/layers-fs/l31-f1.jpg}
}
\subfigure{
\includegraphics[width=.23\textwidth]{images/chapter6/vgg13/layers-fs/l31-f2.jpg}
}
\subfigure{
\includegraphics[width=.23\textwidth]{images/chapter6/vgg13/layers-fs/l31-f3.jpg}
}
\label{fig:vgg13_g_filter_activation_appendix_b}
\end{figure}


\bibliographystyle{IEEEtran}
\bibliography{sections/biblio}


\end{document}